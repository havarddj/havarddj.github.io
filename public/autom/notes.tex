\documentclass[11pt]{report}
\usepackage{amsmath,amsthm,amssymb,amsfonts, mathrsfs, mathtools,
  enumitem, xcolor,bbm, tikz,tikz-cd}
\usepackage[ruled,vlined]{algorithm2e}
\usepackage{physics,hyperref, cleveref}
\usepackage[title]{appendix}

\usepackage[final]{microtype}
\usepackage{float}

\usepackage{tocloft}
\usepackage{etoc}

\tikzcdset{arrow style=tikz}
% \usepackage{newtxmath} 
% \usepackage{lmodern}

\usepackage{CormorantGaramond}
\usepackage{unicode-math}
\setmathfont{Garamond-Math.otf}

% \usepackage{natbib}
\SetKwInput{KwInput}{Input}
\SetKwInput{KwOutput}{Output}

\usepackage[
    backend=biber,
    style=alphabetic,
    natbib=true,
    url=false, 
    doi=true,
    eprint=false,
    backref=true
    ]{biblatex}
% \DeclareFieldFormat{titlecase}{\MakeSentenceCase{#1}}

\addbibresource{/Users/havard/Documents/references.bib}

% \usepackage[margin=1in]{geometry}
\usepackage[right= 2.5cm, left=2.5cm, top= 2.5cm, bottom=1.9cm]{geometry}

% \crefname{algorithm}{alg.}{algs.}
% \Crefname{algorithm}{Algorithm}{Algorithms}

\let\mbb\mathbb
\let\mc\mathcal
\let\mscr\mathscr
\let\mf\mathfrak
\let\mbf\mathbf
\let\ot\leftarrow

\newcommand{\1}{\mathbbm 1}
\newcommand{\N}{\mathbb{N}}
\newcommand{\Z}{\mathbb{Z}}
\newcommand{\Q}{\mathbb{Q}}
\newcommand{\R}{\mathbb{R}}
\newcommand{\G}{\mbb G}
\newcommand{\g}{\mf g}
\newcommand{\F}{\mathbb{F}}
\newcommand{\h}{\mf h}
\newcommand{\C}{\mathbb{C}}
\newcommand{\A}{\mbb A}
\newcommand{\D}{\mscr D}
\newcommand{\p}{\mf p}
\newcommand{\bs}{\backslash}
\renewcommand{\O}{\mc O}
\renewcommand{\L}{\mbb L}
\newcommand{\loc}{\text{loc}}
\newcommand{\nsub}{\trianglelefteq}
\newcommand{\Grp}{\mbf{Grp}}
\newcommand{\Alg}{\mbf{Alg}}
\newcommand{\M}{\mscr M}
% \renewcommand{\N}{\mscr N}
\newcommand{\K}{\mscr K}
\newcommand{\Mod}{\mathbf{Mod}}
\newcommand{\floor}[1]{\left\lfloor #1 \right\rfloor}
% \newcommand{\norm}[2]{\left\lVert #1 \right\rVert_{#2}}
\newcommand{\Norm}[1]{\left\lVert #1 \right\rVert}
\newcommand*{\defeq}{\mathrel{\vcenter{\baselineskip0.5ex \lineskiplimit0pt
      \hbox{\scriptsize.}\hbox{\scriptsize.}}}%
  =}

\newcommand{\ps}[1]{[\! [#1]\!]}
\renewcommand\labelenumi{\textnormal{(\roman{enumi})}}
\renewcommand\theenumi\labelenumi
\renewcommand{\qedsymbol}{$\blacksquare$ \\}
% \renewcommand{\setminus}{\smallsetminus}
\renewcommand{\bar}{\overline}
\renewcommand{\tilde}{\widetilde}
% \renewcommand{\phi}{\varphi}
% \renewcommand{\theta}{\vartheta}
% \renewcommand{\Im}{\mathrm{Im}\,}


\DeclareMathOperator{\supp}{supp}
\DeclareMathOperator{\fin}{fin}
\DeclareMathOperator{\Irr}{Irr}
\DeclareMathOperator{\Sym}{Sym}
\DeclareMathOperator{\Ind}{Ind}
\DeclareMathOperator{\nInd}{nInd}
\DeclareMathOperator{\cInd}{cInd}
\DeclareMathOperator{\cyc}{cyc}
\DeclareMathOperator{\Vol}{Vol}
\DeclareMathOperator{\Ad}{Ad}
\DeclareMathOperator{\St}{St}
\DeclareMathOperator{\sep}{sep}
\DeclareMathOperator{\Pic}{Pic}
\DeclareMathOperator{\rec}{rec}

\DeclareMathOperator{\Spec}{Spec}
\DeclareMathOperator{\cusp}{cusp}

\DeclareMathOperator{\Lie}{Lie}
\DeclareMathOperator{\spn}{span}
\DeclareMathOperator{\sgn}{sgn}

\DeclareMathOperator{\SO}{SO}
\DeclareMathOperator{\Sp}{Sp}
\DeclareMathOperator{\GSp}{GSp}
\DeclareMathOperator{\Sl}{Sl}
\DeclareMathOperator{\SL}{SL}
\DeclareMathOperator{\Gl}{Gl}
\DeclareMathOperator{\GL}{GL}
\DeclareMathOperator{\PGL}{PGL}

\DeclareMathOperator{\Cl}{Cl}
\DeclareMathOperator{\RM}{RM}
\DeclareMathOperator{\Mat}{Mat}
\DeclareMathOperator{\Stab}{Stab}
\DeclareMathOperator{\Id}{Id}
\DeclareMathOperator{\Aut}{Aut}
\DeclareMathOperator{\disc}{disc}
\DeclareMathOperator{\Hom}{Hom}


\DeclareMathOperator{\Qc}{qc}
\DeclareMathOperator{\coh}{coh}
\DeclareMathOperator{\Op}{op}
\DeclareMathOperator{\codim}{codim}

\DeclareMathOperator{\ord}{ord}
\DeclareMathOperator{\Gal}{Gal}
\DeclareMathOperator{\ab}{ab}
\DeclareMathOperator{\Ver}{Ver}
\DeclareMathOperator{\End}{End}

\theoremstyle{plain}
\newtheorem{thm}{Theorem}[section]
\newtheorem*{thm*}{Theorem}

\newcounter{ex}
\setcounter{ex}{0}

\renewcommand{\chaptername}{Part}
\renewcommand{\thechapter}{\Roman{chapter}}
\renewcommand\thesection{\arabic{section}}


\newtheorem{exercise}[ex]{Exercise}

\newtheorem{cor}[thm]{Corollary}
\newtheorem{conj}[thm]{Conjecture}
\newtheorem{lemma}[thm]{Lemma}
\newtheorem{prop}[thm]{Proposition}

\theoremstyle{definition}
\newtheorem{mydef}[thm]{Definition}
\newtheorem{remark}[thm]{Remark}

\newtheorem{example}[thm]{Example}
\newtheorem{non-example}[thm]{Non-example}
\theoremstyle{remark}

\numberwithin{equation}{section}

\newenvironment{problem}[2][Problem]{\it \begin{trivlist}
  \item[\hskip \labelsep {\bfseries #1}\hskip \labelsep {\bfseries
      #2}]}{\end{trivlist}}
\colorlet{phoen}{red!50!black}
\colorlet{purpur}{purple!45!black}
\colorlet{skog}{green!30!black}
\colorlet{hav}{blue!30!black}
\colorlet{himmel}{purpur!90!white}
\hypersetup{
  colorlinks,
  linkcolor={red!50!black},
  citecolor={purple!45!black},
  urlcolor={green!40!black}
}

\begin{document}

\title{Automorphic representations study group}
\author{Håvard Damm-Johnsen}
\date{Hilary term 2023}
\maketitle
\begin{abstract}
  These are (mostly live-\TeX-ed) notes from a study group in Oxford
  from Hilary and Trinity terms 2023. They are quite sketchy, but
  should give an idea of the material covered. Here's
  a
  \href{https://users.ox.ac.uk/~quee4127/automorphic/autom.html}{link
    to the website}.\vspace{-10pt}
\end{abstract}
\tableofcontents

\chapter{Automorphic representations}
\section{Algebraic groups and adeles}
\emph{Speaker: Mick Gielen}

In this talk we will mainly introduce a whole bunch of definitions,
mostly about adeles, a gadget which packages the information about all
the local completions of a number field into one handy topological
ring, and algebraic groups, which are the main characters in the
Langlands programme. According to
the \href{https://mathshistory.st-andrews.ac.uk/Miller/a/}{internet},
the term ``automorphic'' was first used by Klein in the 1890s
to describe functions, now known as automorphic forms, which are
invariant under the action of certain groups. The groups in question
will be subgroups of algebraic groups, and an automorphic
representation is a representation consisting of automorphic forms.
\subsection{Adeles}
\begin{mydef}
  \textbf{Global fields} are finite extensions of $\Q$ or $\F_{q}(x)$,
  that is, number fields or function fields.
\end{mydef}

\begin{mydef}
  A valuation on a field $F$ is a map $v : F \to \R \sqcup \{\infty\}$ satisfying
  for all $a,b \in F$, 
  \begin{enumerate}
  \item $v(a)= \infty$ if and only if $a = 0$,
  \item $v(ab) = v(a) + v(b)$,
  \item $v(a+b) \ge \min(v(a),v(b))$.
  \end{enumerate}
\end{mydef}

\begin{mydef}
  An \textbf{absolute value} is a function $|\,{\cdot} \,| : F \to \R$ satisfying the
  usual axioms (see [Getz, def.~1.2], for example). If $0 < \alpha < 1$ and
  $v$ is any valuation, then $|a|_{v} \defeq \alpha^{v(a)}$ defines an
  absolute value on $F$. 
\end{mydef}

\begin{mydef}
  Two absolute values are \textbf{equivalent} if they induce the same
  topology; a \textbf{place} is an equivalence class of absolute
  values.
\end{mydef}

Places correpsonding to non-archmedean absolute values are called
\textbf{finite}, and the others \textbf{infinite}.

\begin{prop}
  Let $F$ be a global field.
  \begin{enumerate}
  \item If $F$ is a function field, then all places are finite.
    
  \item If $F$ is a number field, then the infinite places are in
    bijection with embeddings $F \hookrightarrow \C$ modulo conjugation, and finite
    places in bijection with prime ideals of $\O_{F}$. Explicitly,
    this is given by
    \begin{equation}
      \label{eq:1}
      \iota : F \hookrightarrow \C \qq{goes to} |x| \defeq |\iota(x)|^{[\iota(F) \otimes \R : \R]},
    \end{equation}
    and
    \begin{equation}
      \label{eq:2}
      \mf p \le \O_{F} \qq{goes to} |x|_{\mf p} \defeq q^{-v_{\mf p}(x)} \text{
        where } q = \# \O_{F}/\varpi \O_{F}
    \end{equation}
    and $v_{\mf p}(x) = \max\{x \in \N : x \in \varpi^{n}\O_{F}\}$. 
  \end{enumerate}
\end{prop}
We define completions in the usual way, as equivalence classes of
Cauchy sequences with respect to the absolute value.

\begin{mydef}
  Let $F$ be a global field. We define the \textbf{adeles over $F$},
  $\A_{F} \defeq \prod_{v}^{'}F_{v}$, where $\prod^{'}$ denotes the restricted
  product,
  \begin{equation}
    \label{eq:3}
    \A_{F} = \{(x_{v})_{v} \in \prod_{v}F_{v} : x_{n} \in \O_{F_{v}} \text{
      for almost all $v$}\}.
  \end{equation}  
\end{mydef}
If $v$ is infinite, we adopt the convention $\O_{F_{v}} = F_{v}$.
The adeles $\A_{F}$ has a natural topology generated by fixing a
finite set of places $S$, and for each $v \in S$ fixing $U_{v}
\subset F_{v}$ open and taking $U = \prod_{v \in S} U_{v} \times \prod_{v \not \in S}
\O_{F_{v}}$. 

\begin{prop}
  $\A_{F}$ is a locally compact Hausdorff topological ring. 
\end{prop}
The diagonal image of $F$ in $\A_{F}$ is discrete.

\begin{mydef}
  Let $S$ be a finite set of places. Then
  \begin{equation}
    \label{eq:4}
    \A^{S}_{F} \defeq \prod_{v \not \in S}^{'} F_{v} \qq{and} \A_{F,S} \defeq
    \prod_{v \in S} F_{v}.
  \end{equation}
  We also set $F_{\infty} = \prod_{v \mid \infty}F_{v}$. 
\end{mydef}

\begin{prop}[Approximation for adeles]
We have a decomposition $\A_{F} = F_{\infty} + \prod_{v \nmid \infty} \O_{F_{v}}+ F$,
where we identify $F$ with its diagonally embedded image. 
\end{prop}


\subsection{Algebraic groups}
\label{sec:algebraic-groups}
We are interested in studying algebraic groups like $\GL_{n}$,
$\SL_{n}$, $\SO_{n}$ etc, which can all be viewed as locally closed
subschemes of $\Mat_{n} \cong \A^{n^{2}}$ (affine $n^{2}$-space, not to be
confused with the adeles.)

\begin{example}
  We can realise the set $\GL_{n}(R)$ as the subset of
  $\A^{n^{2}+1} = \Spec A$ for $A = R [x_{11},x_{12}\ldots, x_{nn},y]$
  given by $\Spec A / (\det(x_{ij})y-1)$. 
\end{example}


\begin{mydef}
An \textbf{affine group scheme} is a functor $G : \Alg_{F} \to \Grp$
represented by an $F$-algebra, denoted $\O(G)$.
\end{mydef}

The goal is to use algebrogeometric methods to study matrix groups.
A morphism of two affine group schemes is given by a natural
transformation of functors, and so we have a category of affine group
schemes over $F$, $\mbf{AffGrpSch}_{F}$.

\begin{remark}
  We define a morphism $H \to G$ to be injective if $\O(G) \to \O(H)$ is
  surjective. If $F$ is a field, then this is equivalent to every
  induced map on $F$-algebras being injective, but not if $F$ is any ring.
\end{remark}
\begin{mydef}
  $G$ is \textbf{linear} if there exists a faithful representation $G
  \hookrightarrow \GL_{n}$ for some $n$.
\end{mydef}

\begin{mydef}
  Suppose $F \hookrightarrow F'$ is a field embedding, and $G$ a group scheme over
  $F$. Then we define the extension of scalars of $G$ to $F'$ by
  $G_{F'}(R) \defeq G(R)$.
\end{mydef}

We can go back as well:
\begin{mydef}
  $\Res_{F}^{F'}G (R) \defeq G(R\otimes_{F}F')$ is called the restriction of scalars.
\end{mydef}
If $F'/F$ is finite and locally free (as an extension of rings), then
the restriction is also linear when $G$ is.

\begin{mydef}
  An affine algebraic group is a group scheme over $F$ represented by
  a finitely generated $F$-algebra.
\end{mydef}

\begin{prop}
  Let $F$ be a topological field. Then there is a natural topology on
  $G(F)$ so that $G(F) \to X(F)$ is continuous for all schemes
  $X/F$. This is compatible with imeersions, fibre products etc. 
\end{prop}

The following shows that we really only need to care about subgroups
of $\GL_{n}$. 
\begin{prop}
If $G$ is an algebraic group, then it is linear.
\end{prop}

An element $x \in \Mat_{n}(\bar F)$ is \emph{semisimple} if it is
diagonalisable over $F$, \emph{nilpotent} if $x^{m}= 0$ for some
$m \in \N$, and \emph{unipotent} if $x-1$ is nilpotent.

Similarly, say $x \in G(\bar F)$ is semisimple (nilpotent, unipotent) if
$\phi(x)$ is semisimple (nilpotent, unipotent) for some faithful representation
$\phi : G \to \GL_{n}$. One can check that this does not depend on $\phi$.

\begin{thm}[Jordan decomposition]
  If $x \in G(\bar F)$, then there exist $x_{s},x_{u}, \in G(\bar F)$
  where $x_{s}$ is semisimple and $x_{u}$ is unipotent such that
  $x = x_{s}x_{u} = x_{u}x_{s}$.
\end{thm}

\begin{mydef}
  The \textbf{Lie algebra} of $G$, $\Lie G$, is the kernel of the map
  \begin{equation}
    \label{eq:5}
    G(F[x]/x^{2}) \to G(F).
  \end{equation}
\end{mydef}
\begin{example}
  Let $G = \GL_{n}$. Then we can find a bijection between $\Lie G$ and
  $\Mat_{n}$  by noting that $(1+ \epsilon A)(1-\epsilon A) = 1$, where $A$ is any matrix.
\end{example}

We define a bracket on $\Lie \GL_{n}$ by $[X,Y] \defeq XY - YX$, and
use this to get brackets on all other linear algebraic groups; note
that $\Lie G \hookrightarrow \Lie \GL_{n}$.

There is natural action of $G$ on $\Lie G$ via conjugation, giving a
map $G \to \GL_{n}(\Lie G)$. This is called the \textbf{adjoint action}. 

We also need the usual algebraic groups $\mbb G_{a}(A) \defeq (A,+)$
and $\mbb G_{m}(A) \defeq (A^{\times},\times)$.
\begin{mydef}
  An algebraic group $T$ is called a \textbf{torus} if $T_{F^{\sep}} \cong
  \G_{m}^{r}$ for some $r \in \N$, which is called the \textbf{rank} of
  $T$.

  If $T \cong \G_{m}^{r}$ without passing to $F^{\sep}$, then $T$ is said
  to be \emph{split}.
\end{mydef}

\begin{mydef}
A character is an element of $X^{*}(G) \defeq \Hom(G,\mbb G_{m})$. 
\end{mydef}
If $G= T$ is a split torus, then $X^{*}(T) \cong \Z^{r}$, but in general
it can be smaller. If $X^{*}(T) = \{0\}$, then $T$ is called
\textbf{anisotropic}.  There is a decomposition
$T = T^{\mathrm{anis}} T^{\mathrm{split}}$, where their intersection
is finite.

\begin{mydef}
  The \textbf{unipotent radical} of $G$, $R_{u}(G)$ is the maximal
  connected (as scheme) unipotent (all elements are unipotent) normal
  (closed) subgroup of $G$. 
\end{mydef}
The radical of a group $H$ is the maximal connected normal solvable
subgroup $H$.

\begin{mydef}
  If $R(G) = \{1\}$ then $G$ is \textbf{semisimple}; if $R_{u}(G)=\{1\}$,
  then $G$ is \textbf{reductive}.
\end{mydef}
Note that $R_{u}(G) \subset R(G)$ so semisimple implies reductive.

\begin{remark}
  We are glossing over some details on \emph{smoothness}, which won't
  be covered here.
\end{remark}
\begin{mydef}
  A \textbf{Borel subgroup of $G$}  is a subgroup $B$ such that
  $B_{F^{\sep}} \subset G_{F^{\sep}}$ is maximal, connected and solvable.
\end{mydef}

These are nice because $G/B$ is always represented by a projective
scheme, and $B$ is minimal with respect to this property.
\begin{mydef}
  A subgroup $P$ of $G$ is \textbf{parabolic} if it contains a Borel
  subgroup of $G$, so that $G/P$ is also projective.
\end{mydef}

\begin{mydef}
A torus $T \subset G$ is a \textbf{maximal torus} if $T_{F^{\sep}}$ is
maximal with respect to inclusion.
\end{mydef}
\begin{example}
  $G = \GL_{n}$, $T = $ diagonal matrices; this forms a split maximal torus.
\end{example}

\begin{prop}
  Reductive groups have maximal torii.
\end{prop}
\begin{mydef}
  We say $G$ is split if a maximal torus is split. If $G$ has a
  Borel subgroup, then it is quasi-split.
\end{mydef}



\begin{example}
  $GL_{n}$ has Borel subgroup given by upper triangular (or lower
  triangular) matrices.
\end{example}

\begin{prop}[Levi decomposition]
  If $P \subset G$ is a parabolic subgroup, then $P = MN$ where $N =
  R_{u}(P)$ and $M \le P$ is a reductive subgroup.
\end{prop}

\section{Hecke algebras and automorphic representations over
  non-Archimedean fields}
\emph{Speaker: Zach Feng}


Let $G$ be a locally profinite group\footnote{So, for any nbhd $U$ of
  the identity there exists an open compact subgroup $K \subset G$ such that
  $e \in K$ and $K \subset U$}
\begin{mydef}
  A rep $(\pi,V)$ is \textbf{smooth} if $\Stab_{\pi}(v) \subset G$ is open for
  all $ v\in V$.
  If $(\pi,V)$ is smooth, then it is \textbf{admissible} if $\dim_{\C}
  V^{U} < \infty$ for all open subgroups $U \subset G$. 
\end{mydef}

Motivation:
Consider $ G \defeq \GL_{n}(\A_{\Q}) = \prod_{p}^{'} \GL_{n}(\Q_{p})$,
where $\GL_{n}(\Q_{\infty}) \defeq \GL_{n}(\R)$. Note that
$\GL_{n}(\Q_{p})$ is locally profinite. If $\pi \colon G \to \GL(V)$ is an
automorphic representation of $G$, then
$\pi = \bigotimes_{p}^{'} \pi_{p}$ where for $p< \infty$, $\pi_{p}$ is smooth and
admissible.

Goal: (i) show that smooth $G$-reps are equivalent to modules of
certain Hecke algebras,
(ii) Spherical Hecke algebras (for $GL_n$).
(iii) Examples for $\GL_{2}$.

\subsection{Hecke algebras}
\label{sec:hecke-algebras}

\begin{mydef}
  Let $\Omega$ be a field. An \textbf{idempotent algebra} is a pair $(H,E)$
  such that $H$ is a (not necessarily unital) $\Omega$-algebra, and $E \subset H$
  is a set of idempotents satisfying:
  \begin{enumerate}
  \item for all $e_{1},e_{2} \in E$, there exists $e_{0} \in E$ such that
    $e_{i}e_{0} = e_{0}e_{i} = e_{i}$ for $i =1,2$,
  \item for all $\phi \in H$, there exists $e \in E$ such that $e\phi = \phi e =
    \phi$. 
  \end{enumerate}
\end{mydef}

\begin{mydef}
If $(H,E)$ is an idempotent algebra and $M$ an $H$-module, then for any
$e \in E$, define $H[e] \defeq eHe$ and $M[e] \defeq eM$. 
\end{mydef}
Note that $M[e]$ is an $H[e]$-module. We say $M$ is \emph{smooth} if
$M = \bigcup_{e \in E} M[e]$, and \emph{admissible} if $\dim_{\Omega}M[e] < \infty$ for
all $e \in E$.

These properties should match up with the corresponding properties for
representations.

\begin{mydef}
  Let $\mc H$ be a set of compactly supported, locally constant
  functions $G \to \C$, and for $\phi_{1},\phi_{2} \in \mc H$ let
  \begin{equation}
    \label{eq:6}
    (\phi_{1} \ast \phi_{2}) (g) = \int_{G} \phi_{1}(gh^{-1})\phi_{2}(h)dh.
  \end{equation}
  This makes $\mc H$ into a $\C$-algebra.
\end{mydef}

Let $K_{0}$ be an open compact subgroup of $G$, and let
\begin{equation}
  \label{eq:7}
  \epsilon_{K_{0}}(g) \defeq \begin{cases}
                        \Vol(K_{0})^{-1} \qq{if} &g \in K_{0}\\
                        0 &\text{ otherwise.}
                      \end{cases}
                    \end{equation}
Then $\epsilon_{K_{0}}$ is an idempotent in $\mc H$. Define $\mc H_{K_{0}}$
to be the subalgebra of $\mc H$ consisting of $K_{0}$-bi-invariant
functions: $f(kgk') = f(g)$ for all $k,k' \in K_{0}$ and $g \in G$; a
simple computation shows that it is closed under
convolution. Moreover, it is unital with identity given by $\epsilon_{K_{0}}$.

\begin{prop}
  We have that $\mc H_{K_{0}} = \epsilon_{K_{0}}\mc H \epsilon_{K_{0}}$. 
\end{prop}
Now let $\phi \in \mc H_{K_{0}}$. Then $\phi$ is constant on sets of the form
$K_{0}gK_{0}$ for $g \in G$, we can write $\phi = \sum_{g} c_{g}
\1_{K_{0}gK_{0}}$ for a finite collection of $g \in G$. For $U \subset G$ open
compact, $K_{0} = U \cap gUg^{-1}$, we know $K_{0}gK_{0} \subset gU$, so every
neighbourhood of $1$ contains a coset of the form $K_{0}gK_{0}$, hence these
form a neighbourhood basis.

Accordingly, $\mc H = \bigcup_{K_{0}} \mc H_{K_{0}}$ so $(\mc H,
\{\epsilon_{K_{0}}\})$ is an idempotent algebra.

Now let $(\pi,V)$ be a smooth $G$-representation. For $\phi \in \mc H$ and $v
\in V$, let $\pi : \mc H \to \GL(V)$ by $\int_{G}\phi(g)\pi(g)vdg$. Then $(\pi,V)$ is
a smooth $\mc H$-module: $V = \bigcup_{K_{0}}\pi(\epsilon_{K_{0}})V$, and each $v $
is fixed by some $K_{0}$.

\begin{thm}
  The category of smooth $G$-representations is equivalent to the
  category of smooth
  $\mc H$-modules.
\end{thm}


Now fix $K \subset G$ open compact, $\pi(\epsilon_{K}) \colon V \to V^{K}$, which is a
$K$-equivariant projection, and $V^{K}$ is a $\mc H_{K}$-module.

\begin{thm}
  Let $(\pi,V)$ be an irreducible smooth $G$-representation, and $K \subset G$
  an open compact subgroup.
  \begin{enumerate}
  \item Either $V^{K} = 0$, or it is simple.
  \item The map sending a smooth irreducible $G$-rep $V$ with
    $V^{K} \ne \{0\}$ to a simple $\mc H_{K}$-module $V^{K}$, is an
    bijection.
  \end{enumerate}
\end{thm}
\begin{proof}[Proof of (i)]
  Let $M \subset V^{K}$ be an $\mc H_{K}$-submodule. Then $\pi(\mc H)M$ is
  $G$-stable, so $\pi(\mc H) M = V$. Now $V^{K} =\pi(\epsilon_{K})V =
  \pi(\epsilon_{K})\pi(\mc H)M = \pi(\epsilon_{K}) \ldots = M $. We leave the proof of (ii) as
  an exercise (or a google search).
\end{proof}

\begin{remark}
  If $V^{K}$ is admissible, then $V$ corresponds to finite-dimensional
  simple $\mc H_{K}$-modules. 
\end{remark}



Let $G = \GL_{n}(F)$, $F/\Q_{p}$ a finite extension with uniformiser
$\varpi$, and $K \defeq \GL_{n}(\O_{F})$. Then $\mc H_{K}$ is called
the \emph{spherical Hecke algebra}.

\begin{thm}[$p$-adic Cartan decomposition]
  $G$ has a decomposition
  \begin{equation}
    \label{eq:8}
    G = \bigsqcup_{e_{1}\ge \ldots e_{n} \in \Z} K\mqty(\varpi^{e_{1}} & & \\ &
    \ddots &\\ & & \varpi^{e_{n}})K
  \end{equation}
\end{thm}

\begin{thm}
$\mc H_{K}$ is commutative.
\end{thm}
\begin{proof}
  Consider the map $x \mapsto x^{t}$ in $\GL(F)$, and let $\sigma$ be the
  endomorphism of $\mc H_{K}$ sending $f$ to $f^{\sigma}(x) =
  f(x^{t})$. Then, by doing some straightforward substitutions, we find 
  \begin{equation}
    \label{eq:9}
    (f_{1}\ast f_{2})^{\sigma}(x) = \int_{G}f_{1}(x^{t}y^{-1})f_{2}(y)dy =
    \int_{G}f_{1}^{\sigma}((y^{t}x)^{-1})f_{2}^{\sigma}(y^{t})dy =
    \int_{G}f_{1}^{\sigma}(y)f_{2}^{\sigma}(xy^{-1})dy  = (f_{2}^{\sigma}\ast f_{1}^{\sigma})(x),
  \end{equation}
  so we are done if we can show that $\sigma = 1$. But $\mc H_{K}$ is
  spanned as a $\C$-vector space by $\1_{K[\varpi^{e_{i}}]K}$ by the Cartan decomposition,
  and these are fixed by $\sigma$. 
\end{proof}

\begin{cor}
  If $\pi$ is a smooth admissible $G$-representation, then $\dim \pi^{K}
  \le 1$.\footnote{Maybe it's enough to require smooth? But not sure.}
\end{cor}
\begin{proof}
  $\pi^{K}$ is a simple $\mc H_{K}$-module, and so is $1$-dimensional if
  it is non-zero.
\end{proof}

\subsection{The case of \texorpdfstring{$\GL_{2}$}{GL2}}
\label{sec:hecke-GL2}

\begin{example}
  Let $G = \GL_{2}(F)$, $K = \GL_{2}(\O_{F})$ as above, fix $B \subset G$
  the upper triangular Borel subgroup. Let $\chi_{1},\chi_{2} \colon F \to
  \C^{\times}$ be characters, and lift to a character on $B$ by $\chi
  \mqty(y_{1}  & \ast \\ 0 & y_{2}) = \chi_{1}(y_{1})\chi_{2}(y_{2})$. Now let
  \begin{equation}
    \label{eq:10}
    \mc B(\chi_{1},\chi_{2}) = \nInd_{B}^{G}\chi \defeq \qty{f : G \to V : f(bg)
    = |a/d|^{1/2}\chi(b)f(g), \qq{and} \exists K_{0} \subset G \text{ open cpt s.t. }
   f(gk_{0}) = f(g) \ \forall k_{0} \in K_{0}}
  \end{equation}
\end{example}
\begin{prop}
  The $G$-representation $\mc B(\chi_{1},\chi_{2})$ is irreducible whenever
  $\chi_{1}\chi_{2}^{-1}(u) \ne |y|^{\pm 1}$.

  If $\chi_{1},\chi_{2}$ are also unramified (i.e. trivial on $\O_{F}^{\times}$), then
  $\mc B(\chi_{1},\chi_{2})^{K} \ne 0$. 
\end{prop}
This is called the \emph{normalised induction}, which is nicer than
the other because
$\mc B(\chi_{1},\chi_{2}) \cong \mc B(\chi_{2},\chi_{1})$.  In the last case, write
$G = BK$ using the Iwasawa decomposition, so that
$\mc B(\chi_{1},\chi_{2})^{K} \C \phi_{K}$, where
$\phi_{K}(b) = |a/d|^{1/2}\chi_{1}(a)\chi_{2}(d)$.

The Hecke algebra has some nice generators for $\GL_{2}$:
$\mc H_{K} = \left\langle T,R,R^{-1} \right\rangle_{\C}$.

\begin{thm}
  Let $\alpha_{i} \defeq \chi_{i}(\varpi)$. Then
  \begin{enumerate}
  \item $T\phi_{K} = q^{1/2}(\alpha_{1}+\alpha_{2})\phi_{K}$,
  \item $R\phi_{K} = \alpha_{1}\alpha_{2}\phi_{K}$.
  \end{enumerate}
\end{thm}
\begin{proof}
  Write $T\phi_{K} = \lambda \phi_{K}$, so that
  \begin{equation}
    \label{eq:11}
    \lambda \phi_{K}(1) = \int_{G}T(g)\phi_{K}(g)dg = \int_{K(...)K}\phi_{K}(g)dg
  \end{equation}
  and decompose as left cosets .... $= |\varpi|^{1/2}\alpha_{2} +
  q|\varpi|^{1/2}\alpha_{2}$, which gives (i).
\end{proof}

If $\pi$ is an irreducible admissible representation of $\GL_{2}(F)$,
then it is one of the following:
\begin{enumerate}
\item $\mc B(\chi_{1},\chi_{2})$ for some $\chi_{i}$ which is irreducible, called the
  \textbf{irreducible principal series},
\item if $\mc B(\chi_{1},\chi_{2})$ is reducible, then the Jordan-H\"older
  decomposition has two factors: a $1$-dimensional
  representation $\chi \circ \det$, and an infinite-dimensional ``special''
  representation,
\item If $\pi$ is not a subquotient of an induced representation, then
  it is \textbf{supercuspidal}.
\end{enumerate}

\section{The Satake isomorphism}
\label{sec:satake}
\emph{Speaker: Håvard Damm-Johnsen}\\ 

\textbf{References:} \cite{cogdell2004}, \cite[\S 2]{getz2010}, 

In this talk, we will introduce the Satake transform, which gives an
isomorphism between a Hecke algebra of an algebraic group and a
corresponding Hecke algebra of a dual group. We will also try to make
this very concrete in the case of $\GL_2$.

\subsection{Root systems for algebraic groups}
\label{sec:root-sys}

Before continuing the study of Hecke algebras over a local field, we
need to review \emph{root systems}, which are fundamental tools in
understanding algebraic groups.

Recall from \cref{sec:algebraic-groups} that a \emph{rank $r$ torus} $T$ of an algebraic group $G$ is a subgroup $T \le G$ such that $T  \cong \G_{m}^{r}$ over an algebraically closed field $\bar k$.

\begin{mydef}
  A \textbf{character} of $T$ is a homomorphism $T \to \G_{m}$. The
  group $X^{*}(T) \defeq \Hom(T,\G_{m})$ is called the
  \textbf{character group of $T$}, and its $\Z$-dual
  $X_{*}(T) \defeq \Hom(X^{*}(T),\Z)$ is called the
  \textbf{cocharacter group of $T$}.
\end{mydef}

Note that since $\G_{m}$ is abelian, a group homomorphism
$\alpha \colon G \to \G_{m}$ will factor through some torus $T \le G$, so we
sometimes call $\alpha$ a character of $G$.

\begin{exercise}
  Check that $X_{*}(T) = \Hom(\G_{m},T)$. 
\end{exercise}

\begin{example}
  If $G = \GL_{2}$, then we have a maximal torus
  $T = \qty{ \mqty(t_{1} & 0 \\ 0 & t_{2})}$, and a character
  $\alpha \colon T \to \G_{m}$ can be written as
  $\alpha(t_{1},t_{2}) = t_{1}^{n_{1}}t_{2}^{n_{2}}$.
\end{example}

Let $\mf g \defeq \Lie G $. In the first talk we defined the
\emph{adjoint action} $\Ad \colon G \to \GL(\mf g)$. Fix a torus
$T \le G$ and consider the restriction $\Ad_{T}$. This gives a commuting
family of operators $\alpha$ acting on $\mf g$, and we can simultaneously
diagonalise these. For each $t \in T$, $\Ad(t) = \alpha(t)$ for some
$\alpha \in X^{*}(T)$, and since each $\alpha$ describes a subspace, there can
only be finitely many non-zero $\alpha$.

\begin{mydef}
The non-zero characters $\alpha$ are called \textbf{roots of $G$ with respect to $T$}, and the finite set of non-zero roots is denoted $\Phi(G,T) \subset X^{*}(T)$.
\end{mydef}

\begin{prop}
  Let $G$ be a connected reductive group, and $T \le G$ a maximal torus with Lie algebra $\mf t$ and roots $\Phi = \Phi(G,T)$.

  \begin{enumerate}
  \item $\mf g = \mf t \oplus \bigoplus_{\alpha \in \Phi} \mf g_{\alpha}$, where $\mf g_{\alpha} \defeq \{ x \in \mf g : \Ad(t)x = \alpha(t)x \text{ for all } t \in T\}$.
  \item For any $\alpha \in \Phi$,
    $T_{\alpha} \defeq (\ker \alpha)^{\circ}$ is a torus of codimension $1$ in $T$. 
  \item For any $\alpha \in \Phi$, there exists a unique
    $\Ad_{T_{\alpha}}$-stable subgroup $U_{\alpha} \le G$, and these are permuted by
    \begin{equation}
      W(G,T) \defeq N_{G}(T)/Z_{G}(T) = \frac{\{ g \in G : gTg^{-1} \subset T \}}{\{g \in G : gtg^{-1} = t \text{ for all } t \in T\}}.
    \end{equation}
    
  \item $G = \left\langle T, \{U_{\alpha}\ : \alpha \in \Phi\} \right\rangle$.
    
  \end{enumerate}
\end{prop}

\begin{mydef}
  The group $U_{\alpha}$ is called the \textbf{root group of
    $\alpha$}, and $W(G,T)$ is called the \textbf{Weyl group}.
\end{mydef}

By duality, there is a natural pairing $X^{*}(T)\times X_{*}(T) \to \Z$. 
\begin{prop}
  Let $\alpha \in \Phi$ be a root. There exists a unique element
  $\alpha^{\vee} \in X_{*}(T)$ satisfying
  $\left\langle \alpha,\alpha^{\vee}\right\rangle =2$.
\end{prop}
\begin{mydef}
  The map $\alpha^{\vee}$ is called the \textbf{coroot of $\alpha$}, and the set of (nonzero) coroots is denoted $\Phi^{\vee}$. 
\end{mydef}
In fact, the map $\alpha \mapsto \alpha^{\vee}$ is injective, so $\# \Phi = \# \Phi^{\vee}$.

\begin{mydef}
  The \textbf{root datum of $(G,T)$} is the tuple $R(G,T)\defeq (X^{*}, \Phi, X_{*},\Phi^{\vee})$. The \textbf{dual root datum} is $(X_{*},\Phi^{\vee},X^{*},\Phi^{\vee})$.
\end{mydef}

An abstract root datum is a tuple of sets $(X^{*},\Phi,X_{*},\Phi^{\vee})$ satisfying certain axioms found in \cite[Def.~2.41]{getz2010}.

\begin{thm}[Chevalley-Demazure]
  A connected reductive group over an algebraically closed field is
  uniquely determined up to isomorphism by its root datum, and any
  abstract root datum gives rise to a connected reductive algebraic group.
\end{thm}
Note that a root datum determines a root system in the sense of Lie algebras, but contains more information: while a root system determines a semisimple Lie algebra, a root datum will also contain information about the centre, so distinguishes $\SL_{n}$ and $\GL_{n}$, for example.

\begin{mydef}
  The \textbf{Langlands dual group}, ${}^{L}G(\C)$, is the complex
  connected reductive group determined by the dual root datum of
  $G$. More generally, if $G$ is a reductive algebraic group over a
  field $k$, then ${}^{L}G$ is the group scheme
  ${}^{L}G \defeq {}^{L}(G \times_{k} \Spec \bar k) \rtimes \Gal(\bar k/k)$.
\end{mydef}

\begin{table}[htbp]
  \centering
  \begin{tabular}{c|c}
    $G$ & ${}^{L}G^{\circ}$ \\ \hline
    $\GL_{n}$ & $\GL_{n}(\C)$ \\
    $\SL_{n}$ & $\PGL_{n}(\C)$ \\
    $\SO_{2n+1}$ & $\Sp_{2n}(\C)$ \\
    $\SO_{2n}$ & $\SO_{2n}(\C)$
  \end{tabular}
  \caption{Table of split algebraic groups and their duals}
  \label{table:dual-groups}
\end{table}

\subsection{The Satake transform}
\label{sec:satake-transform}

A good reference here are these \href{https://people.maths.ox.ac.uk/newton/week5.pdf}{notes by James}!

Fix a reductive algebraic group $G$ over a local field $F$, and a
compact open $K \le G$. Last week, we defined the \emph{Hecke algebra}
$\mc H_{K} \defeq C^{\infty}(K \backslash G / K)$, and explained that a representation $(\pi, V)$ is \emph{unramified}, or \emph{spherical}, if $V^{K_{0}} \ne 0$, when $G = \GL_{n}(F)$ and  $K_{0} = \GL_{n}(\O_{F})$.

We can do the same for a diagonal torus  $T \le G$.
\begin{mydef}
  A representation $\pi \colon T \to \GL(V)$ is \textbf{unramified} if
  $V^{T_{0}} \ne 0$, where $T_{0}\defeq T \cap K_{0}$.
\end{mydef}

If $(\pi,V)$ is irreducible, then $V$ is one-dimensional, so $\pi$ is
actually a character $\alpha \colon \mc H(T/T_{0}) \to \C^{\times}$. But we can
identify $\mc H(T/T_{0})$ with the group of cocharacters $X_{*}(T)$
via $\mc H(T/T_{0}) \cong \C[T/T_{0}] \gets X_{*}(T)$ where the last map is
$\lambda \mapsto \lambda(\varpi)$. In other words, an irrep is precisely an element of
$\Hom_{\Z}(X_{*}(T),\C)$, which is a point on the dual torus $\hat T$, of
the Langlands dual group ${}^{L}G$.

Now let $N$ be the unipotent radical of $G$ -- for $\GL_{n}$ this could be the upper triangular matrices with $1$ along the diagonal --  and consider the map $S \colon \mc H(G,K_{0}) \to \mc H(T,T_{0})$ defined by
\begin{equation}
  \label{eq:satake}
  S(f)(t) = \delta_{B}(t)^{1/2} \int_{N} f(tn)d\mu(n),
\end{equation}
where $\mu$ is the Haar measure on $N$ assigning volume $1$ to
$N \cap K_{0}$.  Here $\delta_{B}$ is the modulus character, i.e.~the
normalising factor which comes from comparing left and right Haar
measures on $B$, and for $\GL_{2}$ we have
$\delta_{B}\mqty(t_{1} & 0 \\ 0 & t_{2}) = |t_{1}/t_{2}|$. One checks that
this satisfies
\begin{equation}
    S(f)(t) = \delta_{B}^{-1/2}(t)\int_{N}f(nt)d\mu(n).
\end{equation}
  
\begin{mydef}
  The map $S$ is called the \textbf{Satake transform}.
\end{mydef}

\begin{example}[$\GL_{2}$, trivial $f$]
  Let's compute the Satake transform of the indicator function of
  $\O_{F}$, $\1_{K_{0}}$. If $t = \mqty(a  & 0 \\ 0 & d)$ then
  \begin{equation}
    \label{eq:satake-trivial}
    S(f) (t) = |a/d|^{1/2} \int_{N}\1_{K_{0}}(tn)d\mu(n) = |a/d|^{1/2}\1_{T_{0}}(t) = \1_{T_{0}}(t).
  \end{equation}
  In other words, $S$ preserves the unit. It's not much harder to see
  that it's an algebra homomorphism.
\end{example}

\begin{exercise}
  Check this, and that the codomain of $S$ is actually
  $\mc H(T,T_{0})$ as claimed.
\end{exercise}

\begin{exercise}
  Check that $S(K_{0}\mqty(\varpi & 0 \\ 0 & 1)K_{0} ) = q^{1/2}$,
  where $q \defeq |\varpi|$.
\end{exercise}
\begin{example}
  Let $\chi \colon T \to \C^{\times}$ be an unramified character, and consider
  $I_{\chi} \defeq \nInd_{B}^{G} \chi$. Since $G = BK_{0} = TNK_{0}$
  (Iwasawa + Levi decomposition) and $\chi$ is unramified, we know that
  $(I_{\chi})^{K_{0}} = \C \cdot v_{\chi}$, where $v_{\chi}$ is called a \emph{spherical
    vector}, acting like $v_{\chi}(tnk_{0}) = \delta^{1/2}(t)\chi(t)$.

  Now $f \in \mc H(G,K_{0})$ acts on $v_{\chi}$ as a scalar
  $\pi_{\chi}(f)$, called the Satake parameter of $\chi$ (I think?). By the
  decomposition above,
  $\mu_{G} = \mu_{T}\times \mu_{N}\times \mu_{K_{0}}$, with Haar measures all normalised to give the intersection with $K_{0}$ volume $1$. We compute
  \begin{equation}
    \label{eq:satake-compu}
    \int_{G}f(g)v_{\chi}(g)d\mu = \int_{T}\int_{N}f(tn) \delta^{1/2}\chi(t)dtdn = \int_{T} S(f)(t)\chi(t)dt,
  \end{equation}
  which can be viewed as evaluating $S(f)$ at the point $\chi$ in $\hat T(\C)$.
\end{example}
This is meant to demonstrate that the Satake transform describes the action of the Hecke algebra on unramified principal series representations.

\begin{thm}
The Satake transform $S$ is an isomorphism onto the subalgebra $\mc H(T,T_{0})^{W}$ consisting of functions invariant under the Weyl group $W$. 
\end{thm}
In particular, the structure of $\mc H(G,K_{0})$ is quite simple; it's the invariants of a polynomial ring $\C[t_{1}^{\pm 1},\ldots, t_{n}^{\pm 1}]$ under a finite group. 

\begin{example}
  Let $G = \GL_{2}$, $F = \Q_{p}$ and $K_{0} = \GL_{2}(\Z_{p})$ as
  above. Then $G \cong \hat G$ (not really, since it's over $\C$, but
  let's ignore that for now) and
  $\hat T = \qty{\mqty(t_{1} & 0 \\ 0 & t_{2}) : t_{i} \in \C^{\times}}$. The
  Weyl group is $S_{2} = C_{2}$, and the nontrivial element acts as conjugation by
  $\mqty(0 & 1 \\ 1 & 0)$, swapping $t_{1}$ and $t_{2}$.  We find that
  $\mc H(G,K_{0}) = \C[t_{1}^{\pm}t_{2}^{\pm}, t_{1}^{\pm} + t_{2}^{\pm}]$.
\end{example}

% What if we take another Hecke algebra? That is, what if we replace $K_{0}$ with some other compact open $K \le G$?

\section{Supercuspidals and the local Langlands Correspondence}
\emph{Speaker: Zach Feng}
\subsection{Supercuspidals}
Let $G$ be a locally profinite group, and $(\pi,V)$ a smooth
representation of $G$.

\begin{mydef}
The \textbf{smooth dual of $V$} is  $V^{\vee} \defeq \bigcup_{K \subset G}(V^{*})^{K}$, where $V^{*} \defeq \Hom_{\C}(V,\C)$. 
\end{mydef}

\begin{prop}
  If $V$ is smooth and admissible, then
  \begin{enumerate}
  \item so is $V^{\vee}$,
    
  \item $V \cong (V^{\vee})^{\vee}$,
  \item if $V$ is irreducible, then so is $V^{\vee}$.
  \end{enumerate}
\end{prop}

For $v \in V$, $\lambda \in V^{\vee}$, define $m_{v,\lambda}\colon G \to \C$ by $m_{v,\lambda}(g) = \lambda(gv)$, the \textbf{matrix coefficient of $v$ and $\lambda$}. 

\begin{mydef}
  A smooth irrreducible representation $(\pi,V)$ is \textbf{supercuspidal} if all matrix coefficients are compactly supported modulo $Z(G)$. 
\end{mydef}

\begin{prop}
  If $V$ is irreducible, then this is equivalent to checking that a single coefficient is compactly supported modulo $Z(G)$.
\end{prop}

How do we find supercuspidal representations? Let $G = \mbb G(F)$ for some reductive group $\mbb G$, $F$ a local field.
\begin{prop}\label{prop:cind-sc}
  Let $H \le G$ be an open subgroup containing $Z$ such that $H/Z$ is compact. Let $(\sigma,W)$ be a finite dimensional irrep of $H$. If
  \begin{equation}
    \cInd_{H}^{G} W \defeq \{f \colon G \to W : f(hg) = \sigma(h)f(g),\
    f\text{ cptly supptd. mod $Z$}\}
\end{equation}
acted on by right translation of $G$ is irreducible and admissible,
then it is supercuspidal.
\end{prop}

\begin{remark}
It is suspected that all supercuspidals arise these way.  
\end{remark}

\begin{thm}[Fintzen]
  If $G$ splits over a tamely ramified extension and $p \nmid \# W$, then
  all supercuspidals arise this way.
\end{thm}

\begin{proof}[Proof of \cref{prop:cind-sc}]
  It suffices to check a single matrix coefficient. By finite-dimensionality of $W$, we can find $w \in W$, $\lambda \in W^{\vee}$ such that $\lambda(w) \ne 0$. Define
  \begin{equation}
    f_{w}(g) \defeq
    \begin{cases}
      \sigma(g)w \text{ if } g \in H, \\ 0 \text{ otherwise,} \\
    \end{cases}
    \qq{and}
    f_{\lambda}(g) \defeq
    \begin{cases}
      \sigma^{*}(g)w \text{ if } g \in H, \\ 0 \text{ otherwise.} \\
    \end{cases}
  \end{equation}
  Now $\left\langle f_{\lambda},f \right\rangle \defeq \left\langle f_{\lambda}(1),f(1) \right\rangle $
  so that $m_{f_{w},f_{\lambda}}(g) = \left\langle \lambda,f_{w}(g) \right\rangle $.
\end{proof}

Let $P = MN$ be the Levi decomposition of a proper parabolic $P$ of
$G$, and $(\pi, V)$ a smooth representation of $G$. Let
$V(N) \defeq \{\pi(n)v - v : v \in V\}$, $V_{N} \defeq V/V(N)$, and
consider $M$ acting on $V_{N}$ by $\pi|_{M} \otimes \delta_{P}^{-1/2}$.

\begin{mydef}
The $M$-module $J_{P}(V) \defeq V_{N}$  is called the \textbf{Jacquet module} of $V$ wrt $P$, and sends smooth representations of $G$ to smooth representations of $V$. 
\end{mydef}

\begin{prop}
  We have
  \begin{enumerate}
  \item $J_{P}$ is exact,
  \item $J_{P}$ preserves admissibility,    
  \item $J_{P}$ is left-adjoint to the parabolic induction functor, $\nInd$.
    In particular, there is a map
    \begin{equation}
      \Hom_{G}(V,\nInd_{P}^{G} W) \xrightarrow{\sim} \Hom_{G}(J_{P}(V),W)
    \end{equation}
  \end{enumerate}
\end{prop}

\begin{thm}[Jacquet]
  A smooth admissible irrep $(\pi,V)$ is supercuspidal if and only if
  $J_{P}(V) = 0$ for all proper parabolics $P \le G$.
\end{thm}

The slogan ``supercuspdial means not coming from parabolic induction'' is formalised by the following:
\begin{thm}
  If $(\pi,V)$ is smooth admissible and irreducible, then there exists a
  parabolic $P = MN$, and an irreducible supercuspidal $(\sigma,W)$ of
  $M$ such that $V$ is a subrepresentation of $\nInd^{G}_{P}W$.
\end{thm}
\begin{proof}
  Since $V$ is irreducible, the statement is equivalent to saying
  there exists a nontrivial $G$-equivariant map
  $V \to \nInd_{P}^{G} W$ for some $P$, $W$. Proceed by induction on
  $\dim G$. For $\dim G = 1$, $G$ is a torus, so every function is
  compactly supported modulo centre. Accordingly, every $\pi$ is
  supercuspidal, and $\pi = \nInd_{G}^{G} \pi$.

  Now suppose $\dim G > 1$. If there are no $W,P$ and $G$-equivariant
  map as above, with $P$ proper, then
  $\Hom(V,\nInd_{P}^{G} W) = 0 = \Hom (J_{P}(V),W)$, so
  $J_{P}(V) = 0$, implying $V$ is supercuspidal.

  Otherwise, pick a proper parabolic $P \le G$ and an admissible
  representation $W$ of $M$, along with a non-zero map
  $V \to \nInd_{P}^{G}W$. By the adjunction, we get a nontrivial map
  $J_{P}(V) \to W$. Because $P$ is proper and $\dim M < \dim G$, we can
  apply the induction hypothesis to $M$: there exists a parabolic
  $P'$ of $M$ with Levi factor $M'$ and supercuspidal $W'$ along with
  a map $W \to \nInd_{P'}^{M}W'$. Composing with $J_{P}(V) \to W$ gives a
  non-zero map $J_{P}(V) \to \nInd_{P'}^{M}W'$, and applying adjunction
  gives $V \to \nInd^{G}_{P}(\nInd^{M}_{P'} W') = \nInd_{W'N}^{G} W'$,
  as required. Finally, by taking an irreducible quotient, we can
  reduce to the case $W'$ irreducible.
\end{proof}

\begin{mydef}[Segments]
Fix $G = \GL_{n}(F)$. 
\begin{enumerate}
\item For any representation $\pi$ of $G$, $s \in \Z$, let
  $\pi(s) \defeq \pi \otimes |\det |^{s}$.
\item A \textbf{segment} is a set of isomorphism classes of
  irreducible supercuspidal representations of $G$ of the form
  $\Delta = \{\pi,\pi(1),\ldots, \pi(r-1)\} =: [\pi,\pi(r-1)]$.
\item Two segments $\Delta_{1}$ and $\Delta_{2}$ are \textbf{linked} if
  $\Delta_{i}\not \subset \Delta_{j}$, and $\Delta_{1} \cup \Delta_{2}$ is also a segment.
\item if $\Delta_{1} = [\pi,\pi']$, $\Delta_{2} = [\pi'',\pi''']$ are segments,
  $\Delta_{1}$ \textbf{precedes} $\Delta_{2}$ if they are linked and $\pi'' = \pi(r)$ for some $r \ge 0$. 
\end{enumerate}
\end{mydef}



\begin{thm}[Bernstein--Zelevinsky classification]
  Let $P = MN$, with $P$ associated to $n_{1} + \ldots + n_{k} = n$ of the form

  \begin{equation}
    \text{(Insert block diagram here)}
  \end{equation}
  \begin{enumerate}
  \item Let $\sigma = \sigma_{1}\otimes \ldots \otimes \sigma_{k}$ where each
    $\sigma_{i}$ is an irreducible supercuspidal. The induction
    $\nInd^{G}_{P} \sigma$ is reducible if and only if there exist
    $i\ne j$ such that $n_{i} = n_{j}$, $\sigma_{i} = \sigma_{j}(1)$. 
  \item Let $m = n_{1} = \ldots = n_{k}$, so that $n = km$. Then
    $\nInd_{P}^{G} \Delta \defeq \bigotimes_{i=0}^{k-1} \nInd_{P}^{G}\pi(i)$ has a unique
    irreducible quotient $Q(\Delta)$.
  \item Consider segments $\{\Delta_{i}\}_{i=1}^{k}$ where each
    $Q(\Delta_{i})$ is an irrep of $G$, and so that $\Delta_{i}$ does not
    precede $\Delta_{j}$ for any $i < j$.Then $\nInd_{P}^{G}(Q(\Delta_{1})\otimes \ldots \otimes Q(\Delta_{k}))$ has a unique irreducible quotient denoted $Q(\Delta_{1},\ldots, \Delta_{k})$. 
  \end{enumerate}
\end{thm}

\begin{example}
  Let $G = \GL_{2}(F)$, $P$ the upper-triangular Borel, and $P =
  MN$. Fix characters $\chi_{1},\chi_{2}$ of $P$. Then the irreducible
  representations of $G$ are:
  \begin{enumerate}
  \item If $\chi_{1}\chi^{-1}_{2} \ne |{\cdot}|^{\pm 1}$, then
    $\nInd(\chi_{1}\otimes \chi_{2})$ is irreducible, and $\Delta_{i} = (\chi_{i})$.
  \item If $\chi_{1}\chi_{2}^{-1} = |{\cdot}|$, let
    $\chi_{2} = \chi |{\cdot}|^{1/2}$ so that $\chi_{2} = \chi |{\cdot}|^{-1/2} $, giving a short exact sequence
    \begin{equation}
      0 \to \chi \boxtimes \St_{G} \to \nInd(\chi_{1}\otimes \chi_{2}) \to \chi \circ \det \to 0,
    \end{equation}
    and $\Delta_{1}$ precedes $\Delta_{2}$;
    $Q(\Delta_{1}\boxtimes \Delta_{2})$ is an irreducible quotient of
    $\nInd(Q(\Delta_{1})\otimes Q(\Delta_{2}))$.
  \item $\chi_{1}\chi_{2}^{-1}= |{\cdot}|^{-1}$, then there is just one segment,
    and $\Delta= (\chi|{\cdot}|^{-1/2}, \chi|{\cdot}|^{1/2})$ and $Q(\Delta) = \chi \boxtimes \St_{G}$.
  \end{enumerate}
\end{example}


\subsection{Local Langlands for \texorpdfstring{$\GL_{n}$}{GLn}}

There exists a unique map $\rec \colon \mc A_{n}(F) \to \mc G_{n}(F)$
where $\mc A_{n}(F)$ is the set of irreducible smooth admissible
representations of $\GL_{n}(F)$, and $\mc G_{n}(F)$ is the set of
$n$-dimensional semisimple complex Weil--Deligne representations of the
Weil--Deligne group $W_{F}$.

This map respects parabolic induction, in the following sense: If
$\Delta = [\pi,\pi(r-1)]$, then
$\rec(Q(\Delta)) = \rec(\pi) \otimes \Sp(r)$, where
$\Sp(n)(z_{1},\ldots,z_{n}) = |z_{i}|^{i}z_{i}$, $N$ is the matrix with
$1$ on the superdiagonal and $0$ elsewhere.\footnote{Tensor product is
  defined differently on RHS!} Here $\Sp(n)$ is the image of the Steinberg representation in the category of the Weil--Deligne representations.
Moreover, $\rec(Q(\Delta_{1}) \boxplus \ldots \boxplus Q(\Delta_{k})) = \bigoplus_{i=1}^{k} \rec(Q(\Delta_{i}))$.

The representations from the previous section have the following images:
\begin{enumerate}
\item $(\chi_{1}\oplus \chi_{2}, 0\otimes 1 + 1 \otimes 0)$
\item $(\chi_{1}|{\cdot}|^{1/2}\oplus \chi_{1}|{\cdot}|^{-1/2}, 0)$, 
\item
  $(\chi_{1}|{\cdot}|^{-1/2}\oplus \chi_{1}|{\cdot}|^{1/2}, \mqty(0 & 1 \\ 0 & 0))$. 
\end{enumerate}

\section{Representations of real reductive groups}
\label{sec:real-groups}
\emph{Speaker: James Newton}


Let $\mbf G$ be a connected reductive group over $\R$; we want to
study real representations of $G \defeq \mbf G(\R)$. Let $K \subset G$ be a
maximal compact subgroup. It turns out that $K = \mbf K(\R)$ for some
algebraic subgroup $\mbf K \subset \mbf G$.

\href{http://virtualmath1.stanford.edu/~conrad/JLseminar/Notes/L6.pdf}{Zhiwei Yun's notes} are a good reference for this talk.

The key example to keep in mind is the following:
\begin{example}
  $\mbf G = \GL_{n}/\R$ and $K = O(n)$.
\end{example}
Also of interest is the following:
\begin{example}
  Let $\mbf G = \Res_{\R}^{\C} \GL_{n} / \C$; $G = \GL_{n}(\C)$ and $K
  = U(n)$. 
\end{example}
For a less trivial example, we can study representations  $U(p,q)$,
the unitary groups of mixed signature. These are not always
quasi-split!

\subsection{Hilbert space representations}

Let $V$ be a Hilbert space with a continuous action of $G$, meaning $G
\times V \to V$ is a continuous map.
\begin{mydef}
  A representation of $G$ is \textbf{unitary} if $G$ is
  norm-preserving, $\| gv\| = \|v\|$ for all $v \in V$ and $g \in G$. 
\end{mydef}

Classical problem: classify unitary representations of $G$.
A key person in the resolution of this was Harish-Chandra, a student
of Dirac.

Less classically, one might think of how
$L^{2}(\SL_{2}(\Z)\backslash \SL_{2}(\R))$ decomposes under the action of
$\SL_{2}(\R)$ by right translation. More generally, we can consider an
adelic quotient
$L^{2}_{\psi}(\GL_{n}(\Q)\backslash \GL_{n}(\A_{\Q}))$; here
$\psi \colon \Q^{\times} \backslash \A_{\Q}^{\times} \to \C^{\times}$ is a unitary central
character, and for $f \in L^{2}$, we require $f(gz) = f(g)\psi(z)$ for
all $z \in \A_{\Q}^{\times}$.

This gives a unitary representation of $\GL_{n}(\R)$, and automorphic
forms show up in these.

Harish-Chandra had the idea that instead of looking at this horribly
big space alone, we can look at how the Lie algebra acts. Let $\g
\defeq (\Lie G)_{\C}$, and let $g_{\R} \defeq \Lie G$.

\begin{mydef}
  Let $V$ be a Hilbert space representation of $G$. Then $v$ is
  \textbf{differentiable} if for all $X \in \g_{\R}$, the limit 
  \begin{equation} 
    \label{eq:12}
    X \cdot  v \defeq \lim_{t\to 0} \frac{\exp(tX)v - v}{t}
  \end{equation}
  exists in $V$. It is \textbf{smooth} if we can iterate this indefinitely. 
\end{mydef}
Now we can extend by linearity to get an action of $\g$ and $U(\g)$,
the universal enveloping algebra. Let $V^{\infty}$ denote the space of
smooth vectors.
\begin{exercise}
  Check that $V^{\infty}$ is $G$-stable.
\end{exercise}
\begin{thm}[Gårding]
  $V^{\infty}$ is dense in $V$. 
\end{thm}
Another way to ``cut down the representation'' is by looking at the
action of the compact subgroup $K$.
\begin{mydef}
  A vector $v \in V$ is \textbf{$K$-finite} if it is contained in a
  finite-dimensional $K$-stable subspace of $V$. 
\end{mydef}
In other words, if $\{k\cdot v : k \in K\}$ spans a finite-dimensional
subspace of $V$. The space of $K$-finite vectors is denoted by
$V^{K-\fin}$.

Note that by using Weyl's averaging trick, $V|_{K}$ is unitarisable,
so $V|_{K} = \hat\bigoplus_{\sigma \in \Irr(K)}V(\sigma)$ where
$V(\sigma) \defeq \sigma \otimes \Hom_{K}(\sigma,V) \hookrightarrow V$ and each
$\sigma$ is finite-dimensional. From this we deduce that $V^{K-\fin} =
\bigoplus_{\sigma}V(\sigma)$. 

\begin{mydef}
  $V$ is \textbf{admissible} if $V(\sigma)$ is finite-dimensional for all
  $\sigma$. 
\end{mydef}
One thing we've lost is that $V^{K-\fin}$ is not $G$-stable in
general.

\begin{prop}
  If $V$ is admissible, then $V^{K-\fin} \subset V^{\infty}$ and stable under the
  action of $U(\g)$. 
\end{prop}
This shows that an admissible $V$ gives a prototypical example of a
$(\g,K)$-module, which we now define:

\subsection{$(\g,K)$-modules}
\label{sec:g-k-modules}

\begin{mydef}
  A \textbf{$(\g,K)$-module} is a complex vector space\footnote{No
    topology specified!} $V$ with 
  ``nice'' compatible actions of $\g$ and $K$, meaning:
  \begin{enumerate}
  \item $V$ is $K$-finite, and for any $v \in V$, the action of $K$ on any
    finite-dimensional subspace $V_{0} \ni v$ is continuous.\footnote{By
    the Peter-Weyl theorem, this implies $K$ acts smoothly.}
  \item $\mf k \defeq \Lie K $ acts on $V$, coinciding with the
    $\g$-action (after extending to $\C$)
  \item if $k \in K$ and $X \in \mf g$, then $k\cdot (Xv) =
    (\Ad(k)X)(kv)$.\footnote{This follows from (ii) when $K$ is a
      connected Lie group.}
  \end{enumerate}
\end{mydef}

\begin{mydef}
  Let $V$ be a $(\g,K)$-module. Then $V$ is
  \textbf{admissible} if $V(\sigma)$ is finite-dimensional for any irrep. $\sigma$ of $K$.
\end{mydef}

In particular, if $V$ is an admissible Hilbert space rep.~ of $G$,
then $V^{K-\fin}$ is an admissible $(\g,K)$-module.

There is a bijection between closed $G$-subrepresentations of $V$ and
sub-$(\g,K)$-modules of $V^{K-\fin}$. Thus, admissible topologically
irreducible $G$-representations map to irreducible $(\g,K)$-modules
(but is not injective; two representations with same $(\g,K)$-modules
are called \emph{infinitesimally equivalent}). Our goal will
eventually be to classify the latter; this gives rise to the Langlands
classification, somewhat analogous to the Bernstein--Zelevinsky
classification.

Another interesting question is: which $(\g,K)$-modules come from
unitary representations? That is, which modules are unitarisable? This
is apparently not fully understood for all $G$.

\begin{prop}[Schur's lemma for $(\g,K)$-modules]
For an irreducible admissible $(\g,K)$-module $V$, $\End(V) = \C$. 
\end{prop}
Let $Z(\g)$ denote the centre of $U(\g)$. Then $Z(\g)$ acts via a
character $\lambda \colon Z(\g)\to \C$, called the \emph{infinitesimal
  character of $V$}.


Let $\mf h \subset \g$ denote the Cartan subalgebra. Then
$\g = \mf n^{+} \oplus \mf h \oplus \mf n^{-}$. Then (fact!)
$Z(\g) \subset U(\mf h)\oplus U(\g)\mf n^{+} \xrightarrow{\mathrm{pr}} U(\mf h)$. 

Let $\rho = 1/2\sum_{\delta \in R_{+}}\delta$ be the sum of the weights of $\mf h$
acting on $\mf n^{+}$, and consider the ``twist'' $t : h \mapsto h-\rho(h)1$. 
\begin{thm}[Harish-Chandra]
  The composite $t\circ\mathrm{pr}$ and the inclusion of
  $U(\g)$ gives a canonical\footnote{Meaning, not dependent on the
    choice of $\mf n^{+}$!} isomorphism
  $\mathrm{HC}\colon Z(\g) \to U(\mf h)^{W}$.
\end{thm}
This can be seen as an archimedean version of the Satake isomorphism.
Therefore, we can think of an infinitesimal character $\lambda \colon Z(\g)
\to \C$ as a $W$-orbit of characters $\mf h \to \C$. 

\begin{example}
  For $G = \GL_{2}(\R)$, $\g = \mf{gl}_{2}(\C)$, then $\mf n^{+}$ is
  the upper triangular (resp. $\mf n^{-}$ the lower triangular
  matrices), $\mf h$ the diagonal matrices. Then $Z(\g) = \C[z,\Delta]$
  where $\Delta$ is the Casimir element and $z = \mqty(1  & 0 \\ 0 & 1)$;
  pick standard basis $\mqty(1  & 0 \\ 0 & -1)$, $e = \mqty(0  & 1 \\
  0 & 0)$ and $f = \mqty(0  & 0 \\ 1 & 0)$. Then
  \begin{equation}
    \label{eq:13}
    \Delta = \frac{1}{2}h^{2}   + fe + ef = \frac{1}{2}h^{2} + h + 2fe \in
    U(\mf h) \oplus U(\g)\mf n^{+}.
  \end{equation}
  Then $\mathrm{pr}(\Delta) = \frac{1}{2}h^{2}+h$, $W = \left\langle \sigma \right\rangle \cong
  C_{2}$, $\sigma h = -h$. Now $\mathrm{pr}(\Delta)$ is not $W$-invariant, but
  twisting sends $h$ to $h-1$ so $\mathrm{HC}(\Delta) =
  \frac{1}{2}(h-1)^{2} + h-1 = \frac{h^{2}-1}{2}$ which is an even
  polynomial, hence invariant under $h \mapsto -h$, which is exactly the
  action of $W$ on $\mf h$. 
\end{example}

\subsection{Classification of irreducible $(\g,K)$-modules}
\textbf{Slogan:} ``Everything is a submodule of a parabolic induction from a 
minimal parabolic''. \footnote{So unlike the non-archimedean case, there are no
supercuspidals!}

Let's restrict our attention to $\SL_{2}(\R)$.\footnote{$GL_{2}(\R)$
  is cleaner to state, but messier to set up; see Yun's notes.} Let $B$ be the
upper-triangular Borel (a minimal parabolic); the \emph{Langlands
  decomposition} is given by $B = MAN$ where $M = \{\pm I\}$,
$A = \qty{\mqty(a & \\ & a^{-1})}$, $a > 0$, and
$N =\qty{\mqty(1 & *\\ & 1)}$. A representation of $B$ is determined
by $\epsilon \in \{0,1\}$ ``mod $2$'' determining $\pm I \mapsto (\pm 1)^{\epsilon}$, $\lambda \in \C$,
determining $a \mapsto a^{\lambda}$, and setting it trivial on $N$. Then
$\big(\Ind_{B}^{G} \epsilon \otimes (\lambda + 1)\big)^{K-\fin}$ is defined to be the
\textit{principal series} $V(\epsilon,\lambda)$, which is a $(\g,K)$-module with
infinitesimal character $h \mapsto \lambda$. These
will be the basic building blocks:

\begin{thm}
  $V(\epsilon,\lambda)$ is irreducible unless:
  \begin{enumerate}
  \item $\lambda \in \Z$, $\epsilon \equiv \lambda +1 \bmod{2}$, in which case
    $\Sym^{n}\C^{2}$ is a subquotient of $V(n \bmod{2}, -n-1)$ and
    $V(n \bmod{2}, n+1)$, and the other JH-constituent is a sum of the
    \emph{holomorphic and anti-holomorphic discrete
      series},
  \item $V(1,0)$, called the \emph{limit of discrete
      series}.
  \end{enumerate}
\end{thm}

When $\lambda \in i\R$, then $V(\epsilon,\lambda)$ is unitary; for $n=0$, $\C$ the trivial
rep of $\SL_{2}(\R)$ is unitary. The discrete series and limit of
discrete series are unitary, but to prove this, one needs to find a
different realisation of the representations.

The \emph{complementary series} are given by $V(\epsilon,\lambda)$,  $\epsilon = 0$, $\lambda \in
(-1,1)$, and these are also unitarisable!

Holomorphic modular forms can be seen as vectors in the holomorphic
discrete series. Maa\ss\ forms give vectors in either (i) the
principal series $V(\epsilon,\lambda)$ for $\lambda \in i\R$, and in particular certain
eigenvalue $1/4$ forms correspond to ``algebraic'' Maa\ss\ forms, or
(ii) limits of discrete series $V(1,0)$. Maass forms with
Laplace-Beltrami eigenvalue in $(0,1/4)$ appear in the complementary
series, but Selberg's $1/4$-conjecture states that no such exist!

\section{Automorphic representations}
\label{sec:autom-repr}
\emph{Speaker: Alex Horawa}\\ 
Let $G$ be a connected reductive group over a number field $F$, $\Sigma$
the set of places of $F$, $\Sigma_{\infty}$ the subset of infinite places. If
$S \subset \Sigma$, then $\A^{S}_{F}$ is the adeles which are $0$ in the
components at places in $S$, and $\A_{F,S}$ its complement in $\A$. 


Recall: $L^{2}_{\psi}(G(F) \backslash G(\A_{F}))$ is an admissible $(\g,K)$-module
with an additional action of $G(\A_{F})$; we write $\A^{\infty}_{F}$ for
the adeles away from infinity. Then the $K$-finite vectors of the
space are a $(\g,K)\times G(\A^{\infty}_{F})$-module.

\begin{mydef}
  An \textbf{automorphic representations} is an admissible
  $(\g,K)\times G(\A^{\infty}_{F})$-module isomorphic to an irreducible subquotient of the
  $K$-finite vectors of $L^{2}_{\psi}(G(F) \backslash G(\A_{F}))$.
\end{mydef}


\subsection{Representations and Hecke algebras}
\label{sec:repr-hecke-algebr}
We give a quick recap of \cref{sec:hecke-algebras}. Let
$C_{c}^{\infty}(G(\A_{F,S}))$ be the set of locally constant functions on
$G(\A_{F,S})$.
Let $\mc H^{S}$ be the Hecke algebra away from $S$, meaning 
the component of places in $S$ is the indicator function of
$\O_{F_{p}}$, or constantly $1$ 

\begin{mydef}
  A representation $(\pi,V)$ of $\mc H^{\infty}$ is \textbf{admissible} if
  for all $K^{\infty}\le G(\A^{\infty}_{F})$ open compact,
  $\pi^{K^{\infty}} = \pi(\1_{K^{\infty}})V$ is finite-dimensional and it is
  \emph{non-degenerate}\footnote{Technical condition I didn't quite
    get}.
\end{mydef}

We can also define a Hecke algebra at $\infty$ as follows:
$G(\R) \defeq (\Res^{F}_{\Q} \mbb G)_{\R}$,
$\mbb G(\R) = G(F\otimes_{\Q} \R)$ is a real reductive group over $\R$, and
fix $K_{\infty}\le G(\R)$.

\begin{mydef}
  The Hecke algebra at $\infty$ is $\mc H_{\infty}\defeq \mc H(G(\R),K_{\infty})$
  the convolution algebra of distributions of $G(\R)$ supported at
  $K_{\infty}$. 
\end{mydef}
Given a representation  $K_{\infty} \to \Aut(V)$ of dimension $n$, we get a
central character and a Haar measure on $K_{\infty}$, $dK_{\infty}$, and can
define an idempotent
\begin{equation}
  \label{eq:14}
\1_{\infty}  = ? 
\end{equation}

\begin{mydef}
  A continuous representation of $K_{\infty}$ on a Hilbert space $V$ is
  \textbf{admissible} if $V$ is an irreducible representation of
  $K_{\infty}$ and $\pi(1_{\infty})V$ is finite-dimensional.
\end{mydef}
Like in the $p$-adic case, a $(\g,K)$-rep is admissible if and only if
the associated $\mc H_{\infty}$-representation is admissible.

\begin{mydef}
  The \textbf{global Hecke algebra} of $G$ is defined to be $\mc H
  \defeq \mc
  H^{\infty}\otimes \mc H_{\infty}$.
\end{mydef}
By definition, a representation $(\pi, V)$ of $\mc H$ corresponds to a
product $(\pi^{\infty},V^{\infty}) \boxtimes (\pi_{\infty},V_{\infty})$. 

\begin{mydef}
  $(\pi,V)$ is \textbf{admissible} if $\pi^{\infty}$ and $\pi_{\infty}$ are both admissible.
\end{mydef}

Note that there is a natural action of $\mc H$ on $L^{2}_{\psi}(G(F) \backslash
G(\A_{F}))$ by convolution. We can use this to get an alternative, equivalent
definition of an automorphic representation:

\begin{mydef}[V2]
  An \textbf{automorphic representation} is an $\mc H$-representation
  which is isomorphic to a subquotient of
  $L^{2}_{\psi}(G(F) \backslash G(\A_{F}))$. 
\end{mydef}

This point of view is more useful to prove the main theorem of the
next section.

\subsection{Flath's factorisation theorem}
\label{sec:flaths-fact-theor}

\begin{thm}[Flath's theorem]
  Let $\pi$ be an automorphic representation. Then for each $v\in \Sigma$ there
  exists a representation $\pi_{v}$ of $G(F_{v})$, such that
  \begin{equation}
    \label{eq:14}
\pi \cong \bigotimes_{v}{}^{\! '}\pi_{v}
  \end{equation} 
\end{thm}
We ought to explain what the prime in the tensor product means; the
following is probably not quite correct.

\begin{mydef}
  Let $I$ be a countable index set, $I_{S}\subset I$ a finite subset, and
  $\{V_{v}\}_{v \in I}$ a collection of $\C$-vector spaces. Let
  $\phi_{v} \in V_{v}$ be a fixed vector. Then
  \begin{equation}
    \label{eq:15}
    W = \bigotimes_{v \in V}{}^{'} \defeq \{(w_{v})_{v} : w_{v} = \phi_{v}
    \text{ for a.e. } v \in I \}.
  \end{equation}
\end{mydef}

Will define
$\bigotimes_{v} {}^{'} V_{v} = \lim_{S \text{ fin}} \bigotimes_{v \in
  S}V_{v}$ for vector spaces. To make this compatible with the algebra structure, we
need the transition maps to be something like
$\phi \mapsto \phi \otimes \phi_{v_{1}}^{0}\otimes \ldots \phi_{v_{n}}^{0}$ where
$\phi_{v_{i}}^{0}$ are idempotents.

Now the central question is, how does this decomposition interact with
group actions?

\begin{example}
Take $\mc H = \mc H^{\infty} \oplus \mc H_{\infty}$. Then $\mc H^{\infty}$ is the
restricted product of $\mc H_{v}$ with respect to the standard
idempotents $\1_{K_{v}}$ where $K_{v}$ is a \emph{hyperspecial
  subgroup}\footnote{Alex doesn't know what this means, so I don't
  need to either.}, for example $G(\O_{F})$.
\end{example}


\begin{mydef}
  A $C_{c}^{\infty}(G(\A^{\infty}_{F}))$-module $W$ is \textbf{factorisable} if
  $W$ is irreducible and $W \cong \bigoplus_{v}{}^{'} W_{v}$ with
  $\phi_{v} \in W_{v}^{K_{v}}$, where each $W_{v}^{K_{v}}$ is
  $1$-dimensional.
\end{mydef}

If this is the case, then up to rescaling the choice of compatible
system is irrelevant.

\begin{thm}[Flath's theorem (V2)]
  If $W$   is an admissible irreducible representation of $\mc H^{\infty}$,
  then $W$ is factorisable. 
\end{thm}

\begin{proof}
  \textbf{Step 1.} ``Weak version'':
  \begin{prop}
    Let $G_{1}$ and $G_{2}$ be locally profinite groups, $G = G_{1}\times G_{2}$.

    \begin{enumerate}
    \item If
   $V_{i}$ is an admissible irreducible representation of $G_{i}$,
   then $V_{1}\otimes V_{2}$ is an admissible irreducible representation of
   $G$.
 \item If $V$ is an admissible irrep of $G$, then there exist irreps
   $V_{1}$ and $V_{2}$ such that $V \cong V_{1} \oplus V_{2}$, and the
   isomorphism class of each $V_{i}$ is determined by $V$. 
    \end{enumerate}

    \begin{proof}[Proof idea]
     Because $V$ is smooth, we can reduce it to a statement about
     Hecke algebras with respect to compact opens, which behave nicely
     with respect to products. 
    \end{proof}
  \end{prop}

  \textbf{Step 2.} The yoga of Gelfand pairs implies:
  \begin{prop}
    Suppose $G$ is unramified outside $S$, for each $v \notin S$, $K_{v} \defeq
    G(\O_{v})$ and $K^{S} = \prod_{v \notin S} K_{v}$.

    If $V^{S}$ is irreducible and admissible then $\dim V^{K^{S}} = 1$.
  \end{prop}

  Now let $W$ be an irreducible admissible representation of
  $C_{c}^{\infty}(G(\A_{F})//G(F)) = \bigotimes_{v} {}^{'} \mc H_{v}$ (or
  whatever).

  Consider $W^{K_{S}}$ is an $A_{S}$-representation where $A_{S} = $
  product of Hecke algebras away from $S$. Then $W^{K^{S}} =
  \bigotimes_{v \in S}W_{1} \otimes W^{S} $ by the first proposition, and by
  the second, $\dim W^{S} = 1$.

  Our goal is to show that $W = \projlim W^{K_{S}}$ where $W^{K_{S}} =
  \bigotimes_{v \in S} W_{v} \otimes W^{S}$, where $W^{S}$ is the spherical
  vectors. (?)
\end{proof}

\subsection{Automorphic multiplicity}
\label{sec:autom-mult}

Let $(\pi,V)$ be an AIR, and pick $\pi_{v}$'s as in Flath's theorem;
suppose further that it is unramified at only finitely many
places. When is $\pi$ an automorphic representation? More generally,
what is the multiplicity of $\pi$ in
$L^{2}_{\psi}(G(\A_{F})/G(F))$?\footnote{At this point, Zach's computer
  ran out of battery, so the rest is a reconstruction from Alex's
  notes.}

\begin{mydef}
  An element $\phi \in L^{2}(G(F)\backslash G(\A))$ is \textbf{cuspidal} if for any parabolic
  $P = MN$ we have
  \begin{equation}
    \label{eq:31}
\int_{N(F)\bs N(\A)}\phi(ng) dn = 0.
  \end{equation}
  The linear subspace of $L^{2}$ consisting of cupsidal automorphic
  representations is denoted $L^{2}_{\cusp}(G(F)\bs G(\A))$.
\end{mydef}

\begin{mydef}
  Let $\pi$ be an admissible irreducible representation of $G(\A)$.
  \begin{enumerate}
  \item The \textbf{multiplicity of $\pi$} is $m(\pi) \defeq \dim
    \Hom_{G(\A)}(\pi,L^{2}_{\cusp}G(F)\bs G(\A))$.
  \item We say $\pi$ is \textbf{equivalent} to $\pi'$ if
    $\pi \cong \pi'$ as $G(\A)$-representaitons.
\item $\pi$ and $\pi'$ are \textbf{weakly equivalent} if $\pi_{v} \cong \pi'_{v}$
  for almost all places $v$ of $F$.
  \end{enumerate}
\end{mydef}

\begin{thm}[Piatetski-Shapiro]
  Let $\pi$ be an automorphic representation of $\GL_{n}(\A)$.
  \begin{enumerate}
  \item (Multiplicity one) $m(\pi) =1$: if $\pi$ and $\pi'$ are equivalent,
    then $\pi = \pi'$.
  \item (Strong multiplicity one) if $\pi$ and $\pi'$ are weakly
    equivalent, then they are isomorphic.
  \end{enumerate}
\end{thm}

We will see in the next lecture that the second statement generalises
the statement that modular eigenform is uniquely determined by
a cofinite set of Hecke eigenvalues.

% \begin{example}[Failure of strong multiplicity one for $G = \GSp_4$]
%   Let $D$ be a quaternion algebra 
% \end{example}

\section{Modular forms and automorphic representations}
\label{sec:modul-forms-autom}
\emph{Speaker: Arun Soor}\footnote{Disclaimer: these notes are an ``expanded version'' of the
  talk, including a lot more words than Arun said. For his (terser)
  notes, see the website.}

In this talk, we will describe how modular forms naturally give rise
to automorphic representations, and how to go back.
 For the rest of the section, let:
\begin{enumerate}
\item $G = \GL_{2}/\Q$
\item $K_{\infty} \le G(\R)$ the maximal compact subgroup, isomorphic to $O(2)$,
\item $Z_{\infty} \le G(\R)$ the centre of $G(\R)$, isomorphic to $\R^{\times} \cong \R^{\times}\mqty(1  & 0 \\ 0 & 1)$.
\end{enumerate}

Let $G(\R)^{+}$ be the connected component of the identity in $G(\R)$,
or equivalently, the set of matrices with positive determinant. For
any subgroup $H \le G(\R)$, let $H^{+} = H \cap G(\R)^{+}$.

\begin{exercise}
Show that $O(n)$ is the maximal compact subgroup by $\GL_{n}(\R)$, by noting that
$O(n)$ is compact, that any compact subgroup of $G(\R)$ fixes some
inner product, hence is conjugate to $O(n)$, and finally that
$gO(n)g^{-1}$ has the same dimension and number of connected
components as $O(n)$.
\end{exercise}

The group $G(\R)^{+}$ acts on $\mf h \defeq \{ z \in \C : \Im z > 0\}$ by
linear fractional transformations:
\begin{equation}
  \label{eq:25}
\mqty(a  & b \\ c & d)\cdot z \defeq \frac{az+b}{cz+d},
\end{equation}
and this is ``almost free''; the only point with a non-trivial
stabiliser is $i$.
\begin{exercise}
  Show that $\Stab i = Z_{\infty}^{+}K_{\infty}^{+}$, where
  \begin{equation}
    \label{eq:41}
Z_{\infty} = \qty{z \mqty(1
& 0 \\ 0 & 1) : z \in \R} \qq{and} K_{\infty}^{+} = \SO_{2}(\R) = \qty{\mqty(\cos\theta
& \sin \theta \\ -\sin \theta  & \cos \theta) : 0 \le  \theta < 2\pi}.
  \end{equation}
\end{exercise}

It follows that we have an equality of sets
$\mf h = G(\R)^{+}/Z_{\infty}^{+}K_{\infty}^{+}$. From this, it is not a stretch
to imagine that we can reinterpret modular forms as functions on
$G(\R)$.

\subsection{Modular forms as automorphic forms}
\label{sec:mfs-aut-fs}

Let $S_{k}(N,\chi)$ denote the modular cusp forms of weight $k$, level
\begin{equation}
  \label{eq:24}
\Gamma_{1}(N) = \qty{\gamma \in \SL_{2}(\Z) : \gamma \equiv  \mqty(1  & * \\ 0 & 1) \bmod{N} }
\end{equation}
and Nebentypus $\chi \colon (\Z/N\Z)^{\times} \to \C^{\times}$; in other words, the
$\C$-vector space of holomorphic functions $f \colon \mf h \to \C$
satisfying
\begin{equation}
  \label{eq:16}
f(\gamma z) = \chi(d)(cz+d)^{k}f(z) \qq{for all} \gamma = \mqty(*  & * \\ c & d) \in \Gamma_{0}(N),
\end{equation}
and which tend to $0$ as $z \to i\infty$. Writing
$f|_{k}\gamma (z) \defeq (cz+d)^{-k}f(\gamma z)$, \cref{eq:16} simply
says $f|_{k}\gamma = \chi(d) f$.


\begin{thm}[Strong approximation for $\SL_2$]
 For any open subgroup $U \le \SL_{2}(\A^{\infty})$, we have $\SL_{2}(\A) =
 \SL_{2}(\Q)\SL_{2}(\R)U$.
\end{thm}
One reference for this is the appendix in \cite{garrett1990} (email me
for a pdf!).

The corresponding statement for $\GL_{1}$ is that
$\A^{\times} = \Q^{\times} \R^{\times}_{+} \hat{\Z}^{\times}$; this is equivalent to the
Chinese remainder theorem.

By combining the two, we get:
\begin{thm}[Strong approximation for $\GL_{2}$]\label{thm:strong-approx-gl2}
  For any open compact $K^{\infty} \subset G(\A^{\infty})$ such that
  $\det(K^{\infty}) = \hat{\Z}^{\times}$ we have $G(\A) = G(\Q)G(\R)^{+}K^{\infty}$.
\end{thm}
It follows that
\begin{equation}
  \label{eq:28}
  G(\Q) \backslash G(\A) / K^{\infty} \cong \Gamma \backslash G(\R)^{+}
\end{equation}
where $\Gamma$ is the image of $G(\Q) \cap G(\R)^{+}K^{\infty}$ in
$G(\R)^{+}$. To be completely explicit, the group $G(\R)^{+}$ acts on
$G(\Q) \backslash G(\A) / K^{\infty}$ by right multiplication in the
$\infty$-component, and
$G(\Q) gK^{\infty}g_{\infty} = G(\Q)gK^{\infty} g'_{\infty}$ if and only if
$g_{\infty}'g_{\infty}^{-1} \in G(\Q) \cap K^{\infty}$. This shows that the map
$\Gamma g_{\infty} \mapsto G(\Q)(1,\ldots,g_{\infty})K^{\infty}$ is injective\footnote{Further
  details can be found
  in
  \href{https://www.math.canterbury.ac.nz/~j.booher/expos/adelic_mod_forms.pdf}{Jeremy
    Booher's notes}}, and it is surjective by
\cref{thm:strong-approx-gl2}.

\begin{example}
  For
  \begin{equation}
    \label{eq:26}
K^{\infty} = K_{0}(N) \defeq \qty{\gamma \in G(\A^{\infty}) \cong G(\hat{\Z}) : \gamma \equiv \mqty(*
  & * \\ 0 & *) \bmod{N}}
  \end{equation}
we get $\Gamma = \Gamma_{0}(N)$. We can similarly define $K_{1}(N)$. 
\end{example}
For $f$ a modular form as above, let $\phi_{f} \in G(\R)^{+} \to \C$ be the
function
\begin{equation}
  \label{eq:16}
  \phi_{f}(g_{\infty}) = f(g_{\infty}i)j(g_{\infty},i)^{-k} \qq{for} g_{\infty} =  \mqty(*  & * \\ c & d)\in G(\R)^{+}
\end{equation}
where $j(g_{\infty},i) \defeq \det(g_{\infty})^{-1/2}(cz+d)$.\footnote{This is
  the convention from \cite{gelbart1973}, but conventions vary!}

\begin{exercise}
  Check that $j(g_{\infty},z)$ satisfies
  \begin{equation}
    \label{eq:26}
    j(g_{\infty}g_{\infty}',z) = j(g_{\infty},g_{\infty}'z)j(g_{\infty}',z) \qq{and} j(z_{\infty}k_{\theta},i) = \sgn(z_{\infty})e^{i\theta},
  \end{equation}
  for $z \in \mf h$, $g_{\infty},g_{\infty}' \in G(\R)$,
  $z_{\infty} \in Z_{\infty}$ and
  $k_{\theta} = \mqty(\cos \theta & -\sin \theta \\ \sin \theta & \cos \theta) \in
  K_{\infty}^{+}$.
\end{exercise}

\begin{exercise}
Extend the slash operator to $g_{\infty} \in G(\R)^{+}$ by $f|_{k}g_{\infty}(z) =
j(g_{\infty},z)^{-k}f(z)$. Then $\phi_{f}(g_{\infty}) = (f|_{k}g_{\infty})(i)$. 
\end{exercise}
Note that we can recover the value of $f$ at $z = x+iy$ from
$\phi_{f}$ by evaluating at
\begin{equation}
  \label{eq:38}
g_{z} \defeq \mqty(y^{1/2} & xy^{1/2} \\ 0& y^{-1/2}):
\end{equation}

indeed, $g_{z}\cdot i = z$, so
$\phi_{f}(g_{z}) = f|_{k}g_{z}(i) = f(z) y^{k/2}$.


We get two transformation laws for $\phi_{f}$: one for
the level (``finite data''):
\begin{equation}
  \label{eq:17}
\phi_{f}(\gamma g_{\infty}) = \phi_{f}(g_{\infty}) \chi(d) \qq{for} \gamma = \mqty(*  & * \\ * & d) \in \Gamma_{0}(N),
\end{equation}
and one for the weight (``infinity data''):
\begin{equation}
  \label{eq:18}
\phi_{f}(g_{\infty}z_{\infty}k_{\theta}) = \phi_{f}(g_{\infty})(\sgn z_{\infty})^{k}e^{-ik\theta}.
\end{equation}
We want to use \cref{eq:28} to turn $\phi_{f}$ into a function on
$G(\A)$. The Nebentypus character $\chi$ only
records data ``at infinity'', but we will use it to define a Hecke character
(i.e.~a continuous character on $\Q^{\times}\backslash \A^{\times}$) $\omega$ as follows:
\begin{enumerate}
\item $\omega$ is trivial on $\Q^{\times}\R_{>0}$.
\item $\omega(d) = \prod_{p\mid N}\omega_{p}(d_{p})$, when
  $d = (d_{p})_{p} \in \A^{\infty}$.
\item $\omega_{p}(d_{p}) \equiv \chi(d_{p})^{-1} \bmod p^{\ord_{p}N}$. 

\item Let $\pi_{p}$ be the image of $p$ under
  $\Q_{p}^{\times} \hookrightarrow \A^{\times}$, so that
  $\pi_{p} = p\cdot 1 \cdot \alpha$ for $\alpha = \pi_{p}/p$ in the decomposition
  $\A^{\times} = \Q^{\times}\R^{\times}\hat{\Z}$. Then
  $\omega(\pi_{p}) = \chi(1/p)^{-1} = \chi(p)$. 
\end{enumerate}
A less concrete way to define $\omega$ is to identify $\Z/N\Z$ with
$\hat{\Z}/N\hat{\Z}$, and lifting to a character on $\hat{\Z}$.

Now we do the usual thing for Nebentype characters: extend to $K_{0}(N)$ by
\begin{equation}
  \label{eq:32}
\omega\mqty(*  & * \\ * & d) \defeq \omega(d) \qq{for} \mqty(*  & * \\ * & d) \in K_{0}(N).
\end{equation}
Note that this is multiplicative.

We can now extend $\phi_{f}$ to $G(\A)$ as follows:
\begin{mydef}
  Fix $f \in S_{k}(N,\chi)$. The \textbf{automorphic form attached to $f$}
  is the function
  \begin{equation}
    \label{eq:33}
    \phi_{f} \colon G(\Q) \backslash G(\A) \to \C
  \end{equation}
  defined by
  $\phi_{f}(g) \defeq \phi_{f}(g_{\infty})\omega(k) = (f|_{k}g_{\infty})(i)\omega(k)$ where by
  \cref{thm:strong-approx-gl2}, $g \in G(\A)$ is written
  $g = \gamma g_{\infty} k$ for $\gamma \in G(\Q)$,
  $g_{\infty} \in G(\R)$ and $k \in K_{0}(N)$.
\end{mydef}
As before, we can recover $f$ by evaluating $\phi_{f}$ at the tuple
$(1,\ldots, g_{z})$ with $g_{z}$ defined in \cref{eq:38}.


\begin{exercise}\label{ex:centre}
  Using the decomposition $\A^{\times} = \Q^{\times}\R^{\times}\hat{\Z}$, check that 
  $\phi_{f}(zg) = \omega(z)\phi_{f}(g)$ for all $g \in G(\A)$.
\end{exercise}
This justifies calling $\omega$ the \textbf{central character of $f$}.
Similarly, we can encode the weight of $\phi_{f}$ by the character
$\sigma_{k}\colon K_{\infty}^{+} \to \C^{\times}$ defined as $\sigma_{k}(k_{\theta}) = e^{-ik\theta}$.

\begin{exercise}
Check that \cref{eq:18} is equivalent to $\phi_{f}(gk_{\theta}) =
\phi_{f}(g)\sigma_{k}(k_{\theta})$ for all $g \in G(\A)$ and $k_{\theta} \in K_{\infty}^{+}$. 
\end{exercise}

\begin{remark}
  If we view $G(\A)$ as acting on $\phi_{f}$ by right multiplication
  (meaning $g \cdot \phi_{f}(g') = \phi(g'g)$), this is equivalent to saying
  $k_{\theta} \cdot \phi_{f} = \sigma_{k}(k_{\theta}) \phi_{f}$, and
  $k\cdot \phi_{f} = \omega(k)\phi_{f}$, for $k \in K_{0}(N)$
\end{remark}

We want to check whether $\phi_{f}$ spans a $(\mf g,K)$-module:
in $\mf g = \mf{gl}_{2}(\C) $ we find
\begin{equation}
  \label{eq:27}
X_{\pm} \defeq \frac{1}{2}\mqty(1 & \pm i \\ \pm i & -1),
\end{equation}
the so-called ``raising and lowering operators'', which more or less
act on $\phi_{f}$ as holomorphic and antiholomorphic derivatives.  The
actions of $\mf g$ and $K_{0}(N)$ are compatible, as
$\Ad(k_{\theta})X_{\pm} = e^{\pm 2i\theta}X_{\pm}$ (Exercise!). Therefore,
\begin{equation}
  \label{eq:34}
 k_{\theta}X_{\pm}\phi_{f} = (k_{\theta}X_{\pm}k_{\theta}^{-1})k_{\theta}\phi_{f} =
 e^{\pm2i\theta}\sigma_{k}(k_{\theta}) \phi_{f} = e^{(k\pm2)i\theta}\phi_{f},
\end{equation}
so the matrices $X_{\pm}$ really raise and lower the weight of
$\phi_{f}$.

\begin{exercise}
  Check the last statement, and also that $X_{-}\phi_{f} = 0$ if and only
  if $f$ is holomorphic.
\end{exercise}

There is another element to account for in $\mf g$: as in \cref{sec:g-k-modules},
$h = \mqty(0 & -i \\ i & 0)$, which gives the Casimir element
\begin{equation}
  \label{eq:19}
\Delta = - \frac{1}{4}h^{2} - \frac{1}{2}(X_{+}X_{-} + X_{-}X_{+})
\end{equation}
in the centre of the universal enveloping algebra of $\mf g$. 
\begin{exercise}
  Check that
  \begin{equation}
    \label{eq:29}
\Delta \phi_{f} = \frac{-k}{2} (\frac{k}{2}-1)\phi_{f},
\end{equation}
and that $Z_{\infty}^{+}$ acts trivially on $\phi_{f}$. 
\end{exercise}

It follows that $Z(\mf g)\phi_{f} = \C[\Delta]\phi_{f}$ is one-dimensional. 

\begin{mydef}
  Let $\omega : \Q^{\times} \bs \A^{\times}\to \C^{\times} $ be a Hecke character of
  conductor dividing $N$, for some $N \in \N$. An
  \textbf{automorphic form} is a function $\phi$ satisfying
\begin{enumerate}
\item $\phi(gk_{\theta}k_{0}) = \sigma_{k}(k_{\theta})\psi(k_{0})\phi(g)$ for all $g \in G(\A)$,
  $k_{\theta} \in K_{\infty}^{+}$ and $k_{0} \in K_{0}(N)$, and
\item $\Delta\phi = \lambda \phi$, for some $\lambda \in \C$. 
\end{enumerate}

The set of automorphic forms is denoted
$A(\psi,\lambda,N,\sigma_{k})$.

\end{mydef}
\begin{mydef}
  An automorphic form $\phi$ is \textbf{cuspidal} if
  \begin{equation}
    \label{eq:35}
    \int_{\Q^{\times}\bs \A^{\times}} \phi(\mqty(1  & x \\ 0 & 1)g)dx = 0 
  \end{equation}
for all $g \in G(\A)$. 
\end{mydef}

\begin{remark}
  Recall that $f \in S_{k}(N,\chi)$ being \emph{cuspidal} means that
  \begin{equation}
    \label{eq:37}
    \int_{0}^{1} f(z+t) dt = 0 \qq{for all} z \in \mf h.
  \end{equation}
  Defining $n_{t} = \mqty(1 & t \\ 0 & 1)$ for $t \in \R$, note that
  $\phi_{f}(n_{t}g_{z}) = f|_{n_{t}}(z) = f(z+t)$. If $\mu_{\infty}$ denotes the
  Haar measure on $\R_{>0}$ with volume $1$, then \cref{eq:37} is
  equivalent to
  \begin{equation}
    \label{eq:39}
    \int_{\R_{>0}}\phi_{f}(n_{t}g_{z})d\mu(t) = 0.
  \end{equation}
  More generally, if $g_{\infty} \in G(\R)$, then by the Iwasawa
  decomposition $G(\R) = B(\R)K_{\infty}^{+}$, write
  $g = g_{z}k_{\theta}$ for $k_{\theta}$ as before and some
  $z \in \mf h$. Then $\phi_{f}(n_{t}g) = \phi_{f}(n_{t}g_{z})e^{ik\theta}$, so 
\begin{equation}
  \label{eq:40}
\int_{\R_{>0}}\phi_{f}(n_{t}g)d\mu(t) = 0
\end{equation}
as well. A consequence of this, along with the fact that adelic
integrals are defined as products of local integrals, is that
cuspidality of $f$ is equivalent to the statement
  \begin{equation}
    \label{eq:36}
    \int_{\Q^{\times}\bs \A^{\times}} \phi_{f}\qty(\mqty(1  & x \\ 0 & 1)g)dx = 0
    \qq{for almost every} g \in G(\A),
  \end{equation}
  where it is understood that we take the normalised Haar measure on
  $\Q^{\times}\bs \A^{\times}$. The ``almost every'' is present in Gelbart, but
  I don't quite know why it's there.
\end{remark}

Let $L^{2}_{\cusp}$ denote the subspace of $L^{2}$ consisting of
functions which are \emph{cuspidal}, that is, which satisfy \cref{eq:35}.

\begin{mydef}
  Let $\psi$ be a Hecke character, fix $\lambda \in \C$ and $N,k \in \N$. 
  A \textbf{cuspidal automorphic form} of weight $k$, level $K_{0}(N)$, spectral
  parameter $\lambda$ and central character $\psi$, is an element $\phi$ of
  $L^{2}_{\cusp}(G(\Q)\bs G(\A),\psi)$ satisfying
\begin{enumerate}
\item $\phi(gk_{\theta}k_{0}) = \sigma_{k}(k_{\theta})\psi(k_{0})\phi(g)$ for all $k_{0} \in
  K_{0}(N)$, $k_{\theta} \in K_{\infty}^{+}$ and $g \in G(\A)$;
\item $\Delta\phi = \lambda\phi$.
\end{enumerate}
The vector space of such functions is denoted
  \begin{equation}
    \label{eq:21}
\mc A_{\cusp}(\psi,\lambda,N,\sigma),
\end{equation}
and is $Z(\mf g)$-finite and $K$-finite. 
\end{mydef}


\begin{prop}
  The assignment $f \mapsto \phi_{f}$ determines an isomorphism of $G(\A)$-modules,
  \begin{equation}
    \label{eq:20}
S_{2}(N,\chi) \to \mc A_{\cusp}(\psi,-\frac{k}{2}(\frac{k}{2}-1),N,\sigma_{k}) \subset
L^{2}_{\cusp}(G(\Q)\backslash G(\A),\psi),
\end{equation}

The Petersson inner product on $S_{k}(n,\chi)$ coincides with the natural
inner product on $L^{2}$.
\end{prop}
This is proved in \cite{gelbart1973}, Prop.~3.1 and 3.2.

\subsection{Modular forms as automorphic representations}
\label{sec:modular-forms-as}
The real power of the adelic theory becomes apparent once we pass from
automorphic forms to their associated representations; then we can
study automorphic forms using the powerful tools of representation
theory. The next three results are the key building blocks in the
theory, telling us that automorphic representations match up with
classical newforms.

As described in \cref{sec:real-groups},
$L^{2}_{\cusp}(G(\Q)\backslash G(\A),\psi)$ decomposes under the action of
$G(\A)$ as a direct sum of irreducible unitary representations.

\begin{thm}[Multiplicity $1$]
  The multiplicity of each representation is $1$.
\end{thm}

\begin{thm}[Strong multiplicity $1$]
  $\pi$ is determined by $\pi_{\infty}$ and a cofinite set of $\pi_{p}$.
\end{thm}
\begin{proof}
  See \cite[Prop.~5.14]{gelbart1973} or \cite{casselman1973}.
\end{proof}

If $p \nmid N$, we have natural Hecke operators $T(p)$ on
$S_{2}(N,\chi)$. On the other hand, elements of
$\mc A_{\cusp}(\psi,\lambda,N,\sigma)$ are fixed by $K_{p}$ for all
$p \nmid N$, and so we have an action of the spherical Hecke algebra,
hence of $\tilde T(p) = \1_{D}$ where
$D = K_{p}\mqty(p & \\ & 1)K_{p}$.

\begin{exercise}
Show that $p^{(k-1)/2}\tilde T(p) = \phi_{T(p)f}$.
\end{exercise}

\begin{prop}
  If $f$ is an eigenform, let $\pi_{f} \defeq G(\A) \phi_{f}$. Then
  $\pi_{f}$ is irreducible.
\end{prop}

\begin{proof}
  By strong multiplicity $1$, it suffices to show any two irreducible
  components $\pi'$ have
  the same local components away from primes dividing $N$. 
  Note that $\pi_{p}'$ is unramified and
  irreducible, classified by its Satake parameter
  $t(\pi_{p})$, a semisimple conjugacy class in $\GL_{2}(\C)$. Write
  \begin{equation}
    \label{eq:22}
t(\pi_{p}') = \mqty(t_{1,p}  &  \\  & t_{2,p}).
  \end{equation}

  Then $\tr t(\pi_{p}') = = t_{1,p} + t_{2,p} = \lambda_{p}$, the
  $T(p)$-eigenvalue of $f$. Now, any irreducible summand of $\pi_{f}$
  contains a function with Hecke eigenvalue $p^{(k-1)/2}\lambda_{p}$. But
  the minimal polynomial of $T_{p}$ is
  $1 - p^{-(k-1)/2}a_{p}X + \chi(p)X^{2} = \det(I - t(\pi'_{p})X)$, so we
  conclude that the Satake parameter of $\pi'_{p}$ is independent of
  choice of $\pi'$, hence $\pi'_{p}= \pi_{p}$. 

  Similarly, at infinity, there's a unique $\pi_{\infty}$ determined by
  $\Delta\phi_{f} = -\frac{k}{2}(\frac{k}{2}-1)$, namely the discrete series
  representation of weight $k$, so $\pi_{\infty}$ is irreducible. 
\end{proof}

As a corollary, we get a map sending a newform $f \in S_{k}(N,\chi)$
to an irreducible constituent of $L^{2}_{0}(G(\Q)\backslash G(\A),\psi)$. Notice
that if $f_{1}$ is old, corresponding to a newform $f_{2}$, then
$\pi_{f_{1}} \cong \pi_{f_{2}}$ by strong multiplicity $1$, so we get a
bijection between newforms and automorphic representations.

Next we define the conductor of $\pi$, and show that it coincides with
the level of $f$ in the case of $\pi = \pi_{f}$. Unsurprisingly, we will
build it from local data: if $\pi \cong \pi_{\infty} \otimes \bigotimes{}^{\! '}\pi_{p}$,
then the \emph{conductor} of $\pi_{p}$,  $c(\pi_{p})$, is the
minimal $p^{r}$ such that
\begin{equation}
  \label{eq:23}
  V_{p}^{K_{0}(p^{r}),\psi} = \qty{v \in V_{p} : \pi_{p}(k)v = \psi(k)v \text{ for
      all } k \in K_{0}(p^{r}) },
\end{equation}
and \emph{global conductor} is $c(\pi) \defeq \prod_{p}c(\pi_{p})$.

\begin{thm}[Casselman]
Each $c(\pi_{p})$ exists, and $\dim V_{p}^{K_{0}(p^{r}),\psi} = 1$.
\end{thm}

Let $N = c(\pi)$. Then for $K_{0}(N) \subset K^{\infty}$, we have
$V^{K_{0}(N),\psi,\sigma_{k}} = \C\phi$ for some $\phi$, depending on
$\pi$, which corresponds to some $\phi_{f}$. Minimality of $N$ implies $f$
is new at $N$. As a result, we have a bijection between newforms $f \in
S_{k}(N,\chi)$ and irreducible constituents of $L^{2}_{0}(G(\Q)\backslash
G(\A),\psi)$ with $c(\pi) = N$ and $\pi_{\infty}$ is discrete series of weight
$k$. For unramified $p$, i.e. $p \nmid N$, this is determined by the
Satake parameters at $p$.

\begin{example}
  If $N$ is squarefree, $\chi$ trivial, then for all
  $p \mid N$, $\pi_{f,p}$ is (a twist by an unramified quadratic
  character of?) the Steinberg representation, because this is the only
  irreducible representation of $\GL_{2}(\Q_{p})$ with conductor $p$.
\end{example}

The following is a consequence of the Weil conjectures:
\begin{thm}[Ramanujan--Petersson conjecture]
  If $f \in S_{k}(N,\chi)$, then $|a_{p}(f)| \le 2p^{(k-1)/2}$ for all  $p \nmid N$.
\end{thm}
This is equivalent to saying $|t_{1,p} + t_{2,p}| \le 2$, hence
$t_{i,p}= p^{s_{i}}$ for $s_{i}\in i\R$. Representations satisfying
analogues of this property are called \emph{tempered
  representations}. Thus we can rephrase the conjecture to the
following:

\begin{thm}[Ramanujan--Petersson conjecture (V2)]
  For any irreducible constituents $\pi$ of $L^{2}_{0}(G(\Q)\backslash G(\A),\psi)$ with
  $\pi_{\infty}$ discrete series of weight $k$, $\pi_{p}$ is
  tempered. 
\end{thm}

This is not known, say, for Maass forms, where $\pi_{\infty}$ is principal series.

More generally, we can consider analogues of this for other reductive groups
than $\GL_{2}$, where it is not generally true: we need to specify
that $\pi$ should be \emph{globally generic}, cf.~ \cite{gan2023}.


\chapter{Tate's thesis}

\section{Topological groups}
\label{sec:topological-groups}


\section{Characters of local fields}
\label{sec:local-fields}
\emph{Speaker: Léo Gratien}\\

In this section, $K$ will denote a local field. As
explained in the previous section, its additive group $K^{+}$ is a
locally compact Hausdorff abelian group, and thus has a Haar measure
$\mu$. For any non-zero $\alpha $ in $K$, we get a new left-invariant measure
$\mu_{\alpha}$ on $K$ by
\begin{equation}
  \label{eq:32}
\int_{K} f(x)\mu_{\alpha}(x) \defeq \int_{K} f(\alpha x)\mu(x).
\end{equation}
By the uniqueness of $\mu$, $\mu_{\alpha} = c(\alpha)\cdot \mu$ for some function $c
\colon F \setminus \{0\} \to \R$.
\begin{exercise}
Check that $c(\alpha)$ defines a norm on $K$.
\end{exercise}

\begin{example}
Let $K = \R$. Then $\1_{[0,\alpha]} = \alpha \1_{[0,1]}$, so $c(\alpha) = |\alpha|$. 
\end{example}

\begin{example}
Let $K = \C$. Then $\alpha \1_{[0,1]^{2}} = \1_{[0,\alpha]^{2}}$, so $c(\alpha) = |\alpha|^{2}$. 
\end{example}

\begin{example}
  Let $K = \Q_{p}$. Since
  $\1_{\Z_{p}} = \sum_{\alpha=0}^{p-1} \1_{\alpha + p\Z_{[p}}$, by translation
  invariance we see that $p\mu(p\Z_{p}) =\mu(\Z_{p})$, so
  $c(p) = p^{-1}$.
\end{example}

In a similar way, one shows: 
\begin{exercise}
Let $K/\Q_{p}$ be a finite extension with uniformiser $\varpi$ and
residue field $F_{q} \cong \O_{K}/\varpi$. Show that $c(p)=  q^{-1}$. 
\end{exercise}

This is an analogue of the following characteristic $p$ result:
\begin{exercise}
Let $K = \F_{q}((x))$. Show that $c(x) = q^{-1}$. 
\end{exercise}

\subsection{Characters of \texorpdfstring{$(K,+)$}{(K,+)}}
\label{sec:characters-k}
We will prove the following in a sequence of lemmas:
\begin{thm}\label{thm:local-dual}
  Let $K$ be a local field, viewed as an additive group. Then $\hat K \cong K$. 
\end{thm}

Note first that $\hat K$ has a non-zero element $\chi_{0}$ since
$\hat{\hat K} \cong K$. Define a map $i : K \to \hat K$ by $i(\alpha) = (x \mapsto
\chi_{0}(\alpha x))$.

\begin{lemma}
The map $i$ is an injective homomorphism, and a homeomorphism onto its
image.   
\end{lemma}

\begin{proof}
  That $i$ is a homomorphism is clear from the definition. It is
  injective because $\chi_{0}$ is nontrivial by assumption. It remains to
  show that $i$ is continuous and open. Recall that the topology on
  $\hat K$ is generated by open sets of the form $U(C,V)$, where $C \subset
  K$ is a compact subset, and $V \subset S^{1}$ is a neighbourhood of
  $1$. Continuity of $i$ means that for any pair $C,V$ there exists $\epsilon >0$,
  such that $B(0,\epsilon) \subset i^{-1}(U(C,V))$. But this follows easily from
  continuity of $\chi_{0}$.

  For openness, we need to show that for any $\epsilon >0$, there exist $C$
  and $V$ such that $i(K) \cap U(C,V) \subset i(B(0,\epsilon))$. {\color{red} I got
    confused here.}
\end{proof}

\begin{lemma}
The image $i(K)$ is dense in $\hat K$.
\end{lemma}

\begin{proof}
  An easy consequence of Pontryagin duality is that there is an
  order-reversing bijection between subsets of $K$ and $\hat
  K$, given by $F \mapsto F^{\perp} \defeq \{\chi \in \hat K : \chi(F) =
  \{1\}\}$. As before, $\{x \in K : i(\alpha)(x) = 1 \, \forall \alpha \in K\} = \{1\}$,
  so $\overline{i(K)}$ corresponds to $\{1\}$.
\end{proof}
By the usual topological argument, \cref{thm:local-dual} follows from
the final lemma:

\begin{lemma}
The image $i(K)$ is closed in $\hat K$.
\end{lemma}

\begin{proof}
  Pick a sequence $(x_{n} = \chi_{0}(\alpha_{n}\cdot)) \in i(K)^{\N}$ such that
  $i(x_{n}) \to \psi$ for some $\psi \in \hat K$. Note that
  $(\alpha_{n})$ is Cauchy by an argument similar to that of openness of
  $i$, so $(\alpha_{n})$ converges to some element $\alpha \in
  F$. We claim that $\psi = \chi_{0}(\alpha \cdot)$. But this follows from continuity of $\chi_{0}$.
\end{proof}

Just like for vector spaces, the different choices of $\chi_{0}$ give
different isomorphisms between $K$ and its dual. However, in our
examples from above, there are certain more or less standard choices:


\begin{example}
Let $K = \R$. Then we can take $\chi_{0}(x) = e^{-2\pi x}$.
\end{example}

\begin{example}
Let $K = \C$. Then we can take $\chi_{0}(z) = e^{-2\pi i \Re(z)}$. 
\end{example}

\begin{example}
Let $K = \Q_{p}$. If $a = \sum_{i \gg -\infty}a_{i}x^{i}$, write $q(a) \defeq a
= \sum_{i < 1}a_{i}x^{i}$ for the ``fractional part'' of $a$. Then
$\chi_{0}(a) = e^{2\pi i q(a)}$ defines an additive character of $K$.
\end{example}


\begin{example}
Let $K/\Q_{p}$ be a finite extension. Then we can take $\chi_{0}(a)
 = e^{2\pi i q(\Tr a)}$ where $q$ is as in the preceding example.
\end{example}

\begin{example}
Let $K = F_{q}((x))$. For $a = \sum_{i \gg -\infty} a_{i}x^{i}$, set $\chi_{0}(a) =
\exp(2\pi i \frac{\Tr a_{-1}}{p})$, where $\Tr$ denotes the trace map
from $\F_{q}$ to $\F_{p}$. 
\end{example}

\subsection{Quasicharacters of \texorpdfstring{$(K^{\times},\times)$}{(K,x)}}
\label{sec:quasichars}

It will be clear in later sections that looking only at \emph{unitary}
characters of $K^{\times}$, meaning characters valued in $S^{1}$ instead of
$\C$, is too restrictive. Therefore we make the
following definition:

\begin{mydef}
Let $K$ be a local field. A \textbf{quasicharacter of $K$} is a
continuous group homomorphism $\chi \colon K^{\times} \to \C^{\times}$. A
quasicharacter $\chi$ is \textbf{unitary} if $|\chi(x)| =1$ for all $x \in
F^{\times}$. 
\end{mydef}
Of course, a unitary quasicharacter is simply a character in the above sense.

\begin{thm}
Any quasicharacter can be decomposed $\chi = \chi_{0} \cdot |{-}|^{s}$ for some
$s\in \C$, where $\chi_{0}$ is unitary. 
\end{thm}
This is not particularly hard: first one shows that the only character
$\chi$ which is unramified (i.e.~trivial on $\O_{F}^{\times}$) is
$x \mapsto |x|^{s}$. Therefore, $\chi \cdot |{-}|^{s}$ will be unitary for some
$s$. Furthermore, $s \in \C$ is unique if $K$ is archimedean, and unique
up to multiples of $2\pi i /\log q$ if the residue field of $K$ has
order $q$.\footnote{What if the residue field is infinite?}

\begin{example}
Let $K = \R$. Then $K^{\times} = \{\pm 1\}\times \R_{>0}$. 
\end{example}

\begin{example}
Let $K = \C$. Then $K^{\times} = S^{1}\times \R_{>0}$. 
\end{example}

\begin{example}
Let $K = \Q_{p}$. Then $K^{\times} = p^{\Z} \cdot \Z_{p}^{\times}$. 
\end{example}

\begin{example}
Let $K/ \Q_{p}$ be a finite extension. Then $K^{\times} = \varpi^{\Z} \times
\O_{F}^{\times}$. 
\end{example}

\begin{exercise}
Decompose $K^{\times}$ when $K = \F_{q}((x))$.
\end{exercise}

\printbibliography%
\end{document}


%%% Local Variables:
%%% mode: latex
%%% TeX-master: t
%%% TeX-master: t
%%% TeX-master: t
%%% TeX-master: t
%%% TeX-master: t
%%% TeX-master: t
%%% End:

