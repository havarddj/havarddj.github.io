\documentclass[11pt,a4paper,leqno]{article}

\usepackage{amsmath,amsthm, amssymb, amsfonts, mathrsfs,
  enumitem,adjustbox,caption,setspace,xspace,listings,pgffor}
\usepackage{lstlang0}           %https://magma.maths.usyd.edu.au/magma/extra/
\usepackage{bbm, rotating, array,
  esint,nicefrac}
\usepackage{xcolor,colortbl}
% Tikz stuff
\usepackage{tikz-cd,tikz}
% \usepackage{pgfplots}


% \usetikzlibrary{patterns}
% \usetikzlibrary{decorations.pathmorphing}
% \usetikzlibrary{decorations.markings}
% \usetikzlibrary{arrows.meta,bending}
% \tikzset{cong/.style={draw=none,edge node={node [sloped, allow upside down, auto=false]{$\cong$}}},
% Isom/.style={draw=none,every to/.append style={edge node={node [sloped, allow upside down, auto=false]{$\cong$}}}}}

% % \usepackage{pst-plot}
% % \usepackage{auto-pst-pdf}

% 
% \usepackage[toc,page]{appendix}

% \usepackage[protrusion=true,expansion=true]{microtype}
\usepackage[
    backend=biber,
    style=alphabetic,
    natbib=true,
    url=false, 
    doi=true,
    eprint=false,
    backref=true
    ]{biblatex}

\addbibresource{/Users/havard/Documents/references.bib}
% \citestyle{apalike}
% \usepackage[nottoc]{tocbibind}
% \usepackage{wrapfig}
\usepackage{float}
\usepackage[stable]{footmisc}
% \usepackage{kpfonts}
% \usepackage[utf8]{inputenc}
% \usepackage[T1]{fontenc}
% \usepackage{yfonts}
% \usepackage{kpfonts,fontspec}

\usepackage{unicode-math}
\setmathfont{Garamond-Math.otf}


% \usepackage[urw-garamond]{mathdesign}
% \usepackage{garamondx}

\tikzcdset{arrow style=tikz}
% \usepackage{newtxmath} 
% \usepackage{lmodern}

\usepackage{CormorantGaramond}
% \setmathfont{Garamond-Math.otf}[StylisticSet={7,9}]


% If we want ebgaramond:
% \usepackage[cmintegrals,cmbraces]{newtxmath}
% \usepackage{ebgaramond}
% \usepackage{ebgaramond-maths}
% \usepackage{tocloft}
% \usepackage{etoc}
\usepackage{physics}
\usepackage[margin=1in]{geometry}
% \usepackage[right= 4.5cm, left=4.5cm, top= 2.5cm, bottom=2.5cm]{geometry}
\usepackage{hyperref}
\usepackage{cleveref}
\definecolor{codegreen}{rgb}{0,0.6,0}
\definecolor{codegray}{rgb}{0.5,0.5,0.5}
\definecolor{codepurple}{rgb}{0.58,0,0.82}
\definecolor{backcolour}{rgb}{0.95,0.95,0.92}

\lstdefinestyle{magma}{
  language=Magma,
  backgroundcolor=\color{backcolour},   
  commentstyle=\color{codegreen},
  keywordstyle=\color{magenta},
  numberstyle=\tiny\color{codegray},
  stringstyle=\color{codepurple},
  basicstyle=\ttfamily\footnotesize,
  breakatwhitespace=false,         
  breaklines=true,                 
  captionpos=b,                    
  keepspaces=true,                 
  numbers=left,                    
  numbersep=5pt,                  
  showspaces=false,                
  showstringspaces=false,
  showtabs=false,                  
  tabsize=2
}

\lstdefinestyle{sage}{
  language=python,
  backgroundcolor=\color{backcolour},   
  commentstyle=\color{codegreen},
  keywordstyle=\color{magenta},
  numberstyle=\tiny\color{codegray},
  stringstyle=\color{codepurple},
  basicstyle=\ttfamily\footnotesize,
  breakatwhitespace=false,         
  breaklines=true,                 
  captionpos=b,                    
  keepspaces=true,                 
  numbers=left,                    
  numbersep=5pt,                  
  showspaces=false,                
  showstringspaces=false,
  showtabs=false,                  
  tabsize=2
}
% \lstset{
%   language=Magma,
% basicstyle=\ttfamily,                   % Code font, Examples: \footnotesize, \ttfamily
% numbers=left,                           % Line nums position
% % numberstyle=\tiny,                      % Line-numbers fonts
% stepnumber=1,                           % Step between two line-numbers
% numbersep=5pt,                          % How far are line-numbers from code
% frame=none,                             % A frame around the code
% tabsize=2,                              % Default tab size
% captionpos=b,                           % Caption-position = bottom
% breaklines=true,                        % Automatic line breaking?
% breakatwhitespace=false,                % Automatic breaks only at whitespace?
% showspaces=false,                       % Dont make spaces visible
% showtabs=false,                         % Dont make tabls visible
% }


% \pagenumbering{gobble} 
\setenumerate{noitemsep}

\let\mbb\mathbb
\let\mscr\mathscr
\let\mc\mathcal
\let\mbf\mathbf
\let\mf\mathfrak
\newcommand{\A}{\mathbb{A}}
\newcommand{\bs}{\backslash}
\newcommand{\1}{\mathbbm{1}}
\newcommand{\N}{\mathbb{N}}
\newcommand{\Z}{\mathbb{Z}}
\newcommand{\Q}{\mathbb{Q}}
\newcommand{\R}{\mathbb{R}}
\newcommand{\h}{\mathfrak{h}}
\newcommand{\G}{\mbb G}
\newcommand{\C}{\mathbb{C}}
\renewcommand{\P}{\mathbb{P}}
% \newcommand{\im}{\mathrm{Im}}
% \newcommand{\re}{\mathrm{Re}}
\renewcommand{\H}{\textfrak{H}}
\renewcommand{\O}{\mathcal{O}}
\newcommand{\ohat}{\widehat{\mathcal{O}}}
\newcommand{\Ca}{\mathscr{C}}
\newcommand{\F}{\mbb{F}}
\renewcommand{\L}{\mathscr{L}}
\newcommand{\p}{\mathfrak{p}}
\newcommand{\Lip}{\mathrm{Lip}\,}
\newcommand{\Jap}[1]{\langle #1 \rangle}
\newcommand{\lav}{\mathrm{Lav}}
\newcommand{\Lav}{\mathrm{Lav}}
\newcommand{\loc}{\text{loc}}
\newcommand{\nsub}{\trianglelefteq}
\newcommand{\Set}{\mathbf{Set}}
\newcommand{\Ab}{\mathbf{Ab}}
\newcommand{\Sc}{\mathbf{Sc}}
\newcommand{\MHS}{\mathbf{MHS}}
\newcommand{\Aff}{\mathbf{Aff}}
\newcommand{\Coh}{\mathbf{Coh}}
\newcommand{\Ring}{\mathbf{Ring}}
\newcommand{\Comp}{\mathbf{Comp}}
\newcommand{\dR}{\mathrm{dR}}
% \newcommand{\Mod}{\mathbf{Mod}}
\newcommand{\shHom}{\mscr{H}\!\!\mscr{om}}
\newcommand{\shExt}{\mscr{Ext}}
\renewcommand{\textnumero}{N\textsuperscript{o}\xspace}

\newcommand{\pow}[1]{[\![#1]\!]}
% \newcommand{\norm}[2]{\left\lVert #1 \right\rVert_{#2}}
\newcommand{\Norm}[1]{\left\lVert #1 \right\rVert}
\newcommand*{\defeq}{\mathrel{\vcenter{\baselineskip0.5ex \lineskiplimit0pt
      \hbox{\scriptsize.}\hbox{\scriptsize.}}}%
  =}

\renewcommand\labelenumi{\textnormal{(\roman{enumi})}}
\renewcommand\theenumi\labelenumi

% \renewcommand{\qedsymbol}{$\blacksquare$ \\} 
\renewcommand{\setminus}{\smallsetminus}
\renewcommand{\bar}{\overline}
\renewcommand{\tilde}{\widetilde}
\renewcommand{\phi}{\varphi}
\renewcommand{\theta}{\vartheta}
\renewcommand{\emptyset}{\varnothing}
\renewcommand{\op}{\text{op}}
\newcommand{\ls}[2]{\qty(\frac{#1}{#2})}
\renewcommand\labelitemi{$\circ$}
% \renewcommand{\arraystretch}{2}
% \renewcommand{\cftsecfont}{\small}

\DeclareMathOperator{\supp}{supp}
\DeclareMathOperator{\Ad}{Ad}
\DeclareMathOperator{\Rep}{Rep}
\DeclareMathOperator{\disc}{disc}
\DeclareMathOperator{\Irr}{Irr}
\DeclareMathOperator{\Ind}{Ind}
\DeclareMathOperator{\ab}{ab}
\DeclareMathOperator{\Vol}{Vol}
\DeclareMathOperator{\nord}{n.ord}
\DeclareMathOperator{\supersing}{ss}
\DeclareMathOperator{\covol}{coVol}
\DeclareMathOperator{\Stab}{Stab}
\DeclareMathOperator{\St}{St}
% \DeclareMathOperator{\ord}{ord}
\DeclareMathOperator{\cd}{cd}
\DeclareMathOperator{\an}{an}
\DeclareMathOperator{\RM}{RM}
\DeclareMathOperator{\sgn}{sgn}
\DeclareMathOperator{\sep}{sep}
\DeclareMathOperator{\old}{old}
\DeclareMathOperator{\new}{new}

\DeclareMathOperator{\res}{res}
\DeclareMathOperator{\alg}{alg}
\DeclareMathOperator{\Alg}{Alg}
\DeclareMathOperator{\gr}{gr}
\DeclareMathOperator{\Zar}{Zar}
\DeclareMathOperator*{\esssup}{ess\,sup}
\DeclareMathOperator{\spn}{span}
\DeclareMathOperator{\SL}{SL}
\DeclareMathOperator{\PSL}{PSL}
\DeclareMathOperator{\Sp}{Sp}
\DeclareMathOperator{\Mp}{Mp}
\DeclareMathOperator{\SO}{SO}
\newcommand{\SpC}{\widetilde{\operatorname{\Sp}}}
\DeclareMathOperator{\GL}{GL}
\DeclareMathOperator{\PGL}{PGL}
\DeclareMathOperator{\pr}{pr}
\DeclareMathOperator{\Id}{Id}
\DeclareMathOperator{\Aut}{Aut}
\DeclareMathOperator{\Hom}{Hom}
\DeclareMathOperator{\Ext}{Ext}
\DeclareMathOperator{\Mod}{Mod}
\DeclareMathOperator{\Gal}{Gal}
\DeclareMathOperator{\Et}{Ét}
\DeclareMathOperator{\et}{ét}
\DeclareMathOperator{\Br}{Br}
\DeclareMathOperator{\Restr}{res}
\DeclareMathOperator{\Op}{Op}
\DeclareMathOperator{\Char}{char}
\DeclareMathOperator{\Div}{Div}
\DeclareMathOperator{\AC}{AC}
\DeclareMathOperator{\OP}{OP}
\DeclareMathOperator{\pSh}{pSh}
\DeclareMathOperator{\Sh}{Sh}
\DeclareMathOperator{\Sch}{Sch}
\DeclareMathOperator{\Spec}{Spec}
\DeclareMathOperator{\Frac}{Frac}
\DeclareMathOperator{\Pic}{Pic}
\DeclareMathOperator{\Jac}{Jac}
\DeclareMathOperator{\Fun}{Fun}
\DeclareMathOperator{\im}{Im}
\DeclareMathOperator{\re}{Re}
\DeclareMathOperator{\coker}{coker}
\DeclareMathOperator{\Cl}{Cl}
\DeclareMathOperator{\Nm}{Nm}
\newcounter{questions}

\theoremstyle{plain}
\newtheorem{question}[questions]{Question}
\newtheorem{theorem}{Theorem}[section]
\newtheorem*{theorem*}{Theorem}
\newtheorem{cor}[theorem]{Corollary}
\newtheorem*{conj}{Conjecture}
\newtheorem{lemma}[theorem]{Lemma}
\newtheorem{prop}[theorem]{Proposition}
\theoremstyle{definition}
\newtheorem{definition}[theorem]{Definition}
\newtheorem{example}[theorem]{Example}
\newtheorem{non-example}[theorem]{Non-example}
\theoremstyle{remark}
\newtheorem*{remark}{Remark}
\newtheorem*{exercise}{Exercise}

\numberwithin{equation}{section}
\setlength{\parskip}{.5em}
% \setlength{\parindent}{16.5pt}

\colorlet{phoen}{red!50!black}
\colorlet{purpur}{purple!45!black}
\colorlet{skog}{green!30!black}
\colorlet{hav}{blue!30!black}
\colorlet{himmel}{purpur!90!white}
\hypersetup{
  colorlinks,
  linkcolor={red!50!black},
  citecolor={purple!45!black},
  urlcolor={green!40!black}
}

% \setcounter{section}{1}
\begin{document}
\title{\vspace{-1cm} The Theta Correspondence}
\author{Håvard Damm-Johnsen}
\date{MT 2023}
\maketitle%
\begin{abstract}
  These are the notes from a study group on the theta correspondence,
  following a set of lecture notes by Wee Teck Gan titled ``The
  Shimura Correspondence à la Waldspurger''. 
\end{abstract}
\begin{spacing}{0.1}
\tableofcontents
\end{spacing}

\section{Introduction \& motivation}
\label{sec:intro}
\emph{Speaker: George Robinson}
\subsection{Half-integer weight modular forms}
\label{sec:half-integer-weight}

We start classically: let $\theta(\tau) = \sum_{n \in \Z}q^{n^{2}}$ where $q =
e^{2\pi i \tau}$, which is natural to study if we want to understand
representations of integers as sums of squares. For example, $n$-th
coefficient of $\theta^{4}$ is the number of ways in which $n$ can be
written as a sum of four squares. This clearly satisfies invariance in
$\tau \mapsto \tau + 1$, but less obviously, the Poisson summation formula implies
that 
\begin{equation}
  \label{eq:1}
\theta(\frac{-1}{4\tau}) = \sqrt{-2i\tau}\theta(\tau).
\end{equation}
We recognise this as the action of the Möbius transformation
associated to $\mqty(0  & 1/2 \\ -2 & 0)$.
One can check that $\left\langle \mqty(0  & 1/2 \\ -2 & 0), \mqty(1  & 1 \\
  0 & 1)\right\rangle \cap \SL_{2}(\Z) = \Gamma_{0}(4)$, and it follows that $\theta$
is ``almost a modular form'', i.e. satisfies $\theta(\gamma \tau) = j(\gamma,\tau)\theta(\tau)$ for any $\gamma = \mqty(*  & *
\\ c & d) \in \Gamma_{0}(4)$, where
\begin{equation}
  \label{eq:2}
  j(\gamma,\tau) = \epsilon_{d}^{-1}
  \qty(\frac{c}{d})(c\tau+d)^{1/2} \qq{and} \epsilon_{d} = \begin{cases}
                                                   1 \text{ if } d \equiv 1 \bmod{4},
                                                   \\ i \text{ if } d \equiv -1
                                                   \bmod{4}.
                                         \end{cases}
\end{equation}
Note that this is a bit different from the usual $j(\gamma,\tau)$, and that
$\theta^{4} \in M_{2}(\Gamma_{0}(4))$. We can take this as a rudimentary
definition of half-integer weight modular form:

\begin{definition}
  Fix an odd integer $\kappa$ . A holomorphic function $f \colon \mf h \to \C$
  is a weight $\kappa/2$ modular form of level $\Gamma \subset \Gamma_{0}(4)$ if for any $\gamma
  \in \Gamma$, $f(\gamma \tau) = j(\gamma,\tau)^{\kappa} f(\tau)$. 
\end{definition}

A better perspective was given by Weil: define a ``toric cover''
$\mc M$ of $\SL_{2}(\R)$ by
\begin{equation}
  \label{eq:3}
\mc M \defeq \{(\gamma, \phi) : \gamma \in \SL_{2}(\R),\ \phi \colon \mf h \to \C \text{
  s.t. } \phi(z)^{2} = (cz+d)t \text{ for some } t \in S^{1}\},
\end{equation}
where $S^{1} = \{z \in \C : |z| = 1\}$. This is naturally an algebraic
group under the operation
$(\gamma,\phi) \cdot (\delta,\psi) = (\gamma \delta, z\mapsto \phi(\delta z)\psi(z))$, and a cover of
$\SL_{2}(\R)$ via $(\gamma,\phi) \mapsto \gamma$, but there is no splitting. For
$\Gamma \le \mc M$ discrete, we can define modular forms to be functions on
the upper half plane $f$ such that $f|_{\kappa}(\gamma,\phi)$, where the slash
action is defined in terms of $j(\gamma,\tau)$. Let
$M_{\kappa/2}(\Gamma)$ denote the vector space of all such functions, and
$S_{\kappa/2}(\Gamma)$ be the space of cusp forms, i.e. those whose first
Fourier coefficient vanishes.

If $\Gamma \subset \Gamma_{0}(4)$, then there is a natural splitting
$\gamma \mapsto (\gamma, j(\gamma,\cdot))$ which recovers the above notion. In fact, it
suffices to pass to a subgroup of $\mc M$ which is a double cover of
$\SL_{2}(\R)$ (as opposed to a toric extension), denoted
$\SpC_{2}(\R)$, and half-integer weight modular forms are
the natural automorphic forms on this group.

\begin{remark}
  One can show that $\SpC_{2}(\R)$ is not a linear
  algebraic group, by arguing that any finite-dimensional faithful
  representation (i.e. injection) $\SpC_{2}(\R) \to \GL(V)$
  factors through $\SL_{2}(\R)$, hence cannot be injective. 
\end{remark}
\subsection{Fourier coefficients}
\label{sec:four-coeffs}
To understand the Fourier coefficients of modular forms, we usually
introduce Hecke operators, and this works in the half-integer weight
case as well. Let $\Gamma \subset \SpC_{2}(\Z)$ be a subgroup of finite index. Given
$\zeta \in \GL_{2}(\Q)$, we can decompose the double cosets as $\Gamma \zeta \Gamma =
\bigsqcup_{i}\Gamma \zeta_{i}$ for some $\zeta_{i} \in \SpC_{2}(\R)$, and define
\begin{equation}
  \label{eq:4} 
f |_{\kappa}[\Gamma \zeta \Gamma] \defeq \sum_{i} f|_{\kappa}\zeta_{i}.
\end{equation}
Just like for classical modular forms, this gives an endomorphism of $M_{\kappa/2}(\Gamma)$.
\begin{theorem}[Shimura]
Let $N$ be a positive integer divisible by $4$, and let $\Gamma =
\Gamma_{0}(N)$. Then for each $m \in \N$ there is a Hecke operator $T(m) =
T_{\kappa,\chi}^{N}(m)$ acting on $M_{\kappa/2}(\Gamma,\chi)$ which maps cusp forms to cusp
forms. Furthermore, if $(m,m') = 1$, then $T(m)$ and $T(m')$ commute.
\end{theorem}
However, something strange happens in this theory:
\[ \text{if $m$ is not a perfect square, then } T(m) = 0. 
\]
Let us look at a concrete example:
\begin{example}\label{eg:9/2-mf}
  Let $\kappa = 9$ and $N=4$. The space $S_{9/2}(\Gamma_{0}(4))$ is
  one-dimensional, spanned by 
  \begin{equation}
    \label{eq:4}
f(\tau) = \frac{\eta(2\tau)^{12}}{\theta(\tau)^{3}} = q - 6q^2 + 12q^3 - 8q^4 + 12q^6 - 48q^7 + 48q^8 - 15q^9 + 60q^{10} - 12q^{11} + \ldots
  \end{equation}
  where $\eta$ denotes the Dedekind eta function. By Shimura's theorem,
  $f$ is an eigenfunction of all the Hecke operators, and the
  computation of all its Fourier coefficients reduces to computing
  \begin{enumerate}
  \item The eigenvalues of $T(p^{2})$ for $p$ prime, and 
  \item the coefficients $a_{n}(f)$ for squarefree integers $n$.
  \end{enumerate}
\end{example}

\subsection{The Shimura correspondence}
\label{sec:shim-corr}
Fix an eigenform $f = \sum_{n=1}^{\infty}a_{n}q^{n} \in S_{\kappa/2}(\Gamma_{0}(N),\chi)$
with $T(p^{2})$-eigenvalue $\omega_{p}$, and for $t \in \N$ fixed,
one computes
\begin{equation}
  \label{eq:5}
\sum_{n=1}^{\infty}\frac{a_{tn^{2}}}{n^{s}} = a_{t} \prod_{p}(1 -
\chi(p)\qty(\frac{-1}{p})^{(\kappa-1)/2}\cdot \qty(\frac{t}{p})\cdot p^{(\kappa-3)/2}) \cdot (1 -
\omega_{p}p^{-s} + \chi(p)^{2}p^{\kappa-2-2s})^{-1}.
\end{equation}
Shimura observed that the last factor in the product is
independent of $t$, and therefore defined
\begin{equation}
  \label{eq:5}
L(s,f) \defeq \prod_{p}(1 - \omega_{p}p^{-s} + \chi(p)^{2}p^{\kappa-2-2s})^{-1}
\end{equation}
to be the $L$-function of $f$. But the shape of this corresponds
exactly to the $L$-function of a \emph{classical} modular form $F$
of weight $\kappa-1$, Nebentype $\chi^{2}$ and $T(p)$-eigenvalue $\omega_{p}$.
Using Weil's converse theorem, Shimura then proved:

\begin{theorem}[Shimura Correspondence]
Let $\kappa \ge 3$, and let $f \in S_{\kappa/2}(\Gamma_{0}(N),\chi)$ be as above. If $L(s,f)
= \sum_{n=1}^{\infty}\frac{A_{n}}{n^{s}}$, then the formal power series $F(\tau)
= \sum_{n=1}^{\infty} A_{n}q^{n}$ defines an element of $M_{\kappa-1}(\Gamma_{0}(N/2),\chi^{2})$.
\end{theorem}
The form $F$ is called the \textbf{Shimura lift} of $f$.
\begin{example}
  For $\kappa = 9$ as above, one computes that $S_{8}(\Gamma_{0}(2))$ is
  one-dimensional, spanned by the form $F$ with $q$-expansion
  \begin{equation}
    \label{eq:7}
q - 8q^2 + 12q^3 + 64q^4 - 210q^5 - 96q^6 + 1016q^7 - 512q^8 - 2043q^9 + 1680q^{10} + 1092q^{11} + \ldots
  \end{equation}
which is the Shimura lift of the form in \cref{eg:9/2-mf}.
\end{example}
In a sense, Shimura's proof is unsatisfying because it appeals to a
global result (Weil's converse theorem, which roughly says that
modular forms are determined by twists of its $L$-functions, which are
global objects) to prove a local results, namely the equality of the
Hecke eigenvalues $T(p^{2})f = T(p)F$, which can be defined purely
locally.

The goal of the next few lectures is to develop the machinery to
understand what is going on locally in Shimura's correspondence. In
particular, we will pass to the language of automorphic forms and
representations, details of which can be found in the
\href{https://users.ox.ac.uk/~quee4127/automorphic/autom.html}{notes
  from last term's study group}, or a book like Gelbart's, \cite{gelbart1973}.

Later on, we will define an adelic group $\SpC_{2}(\A)$, and
upgrade a half-integer weight modular form $f$ to a function
$\phi_{f}\colon\Sp_{2}(\Q) \backslash \SpC_{2}(\A) \to \C$ spanning an automorphic
representation $\pi_{f}$. By a version of Flath's
theorem, $\pi_{f} = \otimes^{'}\pi_{f,v}$ where each $\pi_{f,v}$ is an irreducible
admissible representation of $\SpC_{2}(\Q_{v})$ Our goal is to understand the
relationship between the local representations $\pi_{f,v}$ and $\pi_{F,v}$
coming from the Shimura lift $F$. This is essentially what Waldspurger
did.

To motivate Waldspurger's (or perhaps Weil's) idea, it is helpful to go forward in time to Niwa
\cite{niwa1975}, who proved:
\begin{theorem}[Niwa]
  Shimura's correspondence can be realised as integrating against a
  kernel function:
  \begin{equation}
    \label{eq:9}
    F(\tau) = \int_{\Gamma_{0}(N)\backslash \mf h} \Theta_{\Phi}(z,\tau)\overline{f(z)}dz,
  \end{equation}
where $\Theta(z,\tau)$ is some explicit theta function, roughly of the shape
$\sum_{(a,b,c)\in \Z^{3}} q^{ac-b^{2}}$.
\end{theorem}
What is this theta function? A key idea, perhaps due to Weil, is that
theta functions such as $\Theta_{\Phi}$, can be ``explained'' in terms of the
\emph{Weil representation}, a distinguished infinite-dimensional
representation of groups like $\SpC_{2}(\R)$.

\begin{example}
  Let $V$ be an orthogonal space of dimension $m$, and let $W$ be a
  symplectic space of dimension $2n$.\footnote{These terms, and the
    example, will be studied more carefully in the next lecture.} Then
  there is a natural map
  $O(V)\times \SpC(W) \hookrightarrow\SpC(V \otimes W)$, and $\SpC$ has a ``nice''
  representation\footnote{This will be the Weil representation.}
  $\omega_{\psi} = \omega_{V,W,\psi} \colon \SpC(V \otimes W) \to \mc S(V^{n})$. This gives an
  explicit map
  $\Theta \colon \mc S(V^{n}) \to \mscr A(\SpC(V\otimes W))$, by
  $\Theta(\Phi)(g) = \sum_{v \in V^{n}(\Q)}(\omega_{\psi}(g)\cdot \Phi)(v)$.

  For Niwa, $V$ is the $3$-dimensional orthogonal space with quadratic
  form $ac -b^{2}$, and $W$ is a $2$-dimensional symplectic space, so 
  $\SpC(W) \cong \SpC_{2}(\R)$. Integrating against $\overline{f(z)}$ can
  be thought of as picking out the $f$-isotypic
  component\footnote{I.e. $f$-eigenspace} inside $L^{2}(O(V)\times
  \SpC(W))$. The connection with classical modular forms comes from
  noting that $O(V)(\R) \cong \SL_{2}(\R)$, by an accidental isomorphism.
\end{example}

In general, the goal will be to use the Weil representation to 
understand irreducible representations of $\SpC(W)$ via decomposition
\begin{equation}
  \label{eq:10}
\Irr(\SO(V^{+})) \sqcup \Irr(\SO(V^{-})) \leftrightarrow \Irr(\SpC(W)),
\end{equation}
for suitable subspaces $V^{\pm}$, and hope that this is some kind of
bijection.

\begin{theorem}[Waldspurger]
  We have a decomposition
  \begin{equation}
    \label{eq:11}
L^{2}_{\disc}(\SpC(W)) = \qty(\text{easy theta functions}) \oplus
\qty(\bigoplus_{\pi \text{ on } \SO(V^{+})} L^{2}_{\pi}) \oplus \qty(\bigoplus_{\pi \text{ on } \SO(V^{-})} L^{2}_{\pi}).
  \end{equation}
\end{theorem}
I'm not quite sure if this is specific to the $V$ and $W$ we chose
already, or if it's a general thing.

A natural question to ask is, when is $\Theta(f)$ non-zero? Often it
suffices to compute a Fourier coefficient, or more generally, a
\emph{period}. Let $F$ be as above, and $C \subset \SpC(W)$. Then by a
change of variables,
\begin{equation}
  \label{eq:12}
  \int_{C}F(\tau)d\tau = \int_{C}\int_{\Gamma_{0}\backslash \mf h} \Theta_{\Phi}(z,\tau)\overline{f(z)} dzd\tau =
  \int_{\Gamma_{0}(N)\backslash \mf h}\overline{f(z)} \int_{C}\Theta_{\Phi}(z,\tau)dzd\tau,
\end{equation}
and with a good choice of $C$ we might realise the inner integral as a
theta lift of the constant function. Traditionally, the theta lift of
a constant function becomes an Eisenstein series, and a theorem
proving such a result is referred to as a \emph{Siegel--Weil theorem}.

If we are in such a situation, then
\begin{equation}
  \label{eq:13}
\eqref{eq:12} = \int_{Y_{0}(N)} \overline{f(z)}E(z,s)dz =
\text{Rankin--Selberg $L$-value}.
\end{equation}
If we know a non-vanishing result for such $L$-values, then we can
deduce that $\Theta(f)\neq 0$.

\section{The metaplectic group and the Weil representation}
\label{sec:metaplectic-weil}
\emph{Speaker: James Newton}


We start out locally: let $k$ be a local field of characteristic $0$.
The origins of the Weil representation, or the ``Segal--Shale--Weil
representation'', lie in quantum mechanics, see for example
\href{https://www-users.cse.umn.edu/~garrett/m/v/oscillator_repn.pdf}{these
  notes by Garrett}.
\subsection{Symplectic vector spaces}
\label{sec:sympl-vect-spac}

\begin{definition}
  A \textbf{symplectic space} is a $k$-vector space $W$ equipped with a
  non-degenerate alternating bilinear form
  $\left\langle {-},{-} \right\rangle W \times W \to k$, also known as a \emph{symplectic
    form}.
\end{definition}

\begin{example}\label{eg:standard-sympl}
  The easiest example of a symplectic space is $k\oplus k$ with standard
  basis $e_{1},e_{2}$, and the standard symplectic form
  $\left\langle e_{1},e_{2} \right\rangle= 1$, $\left\langle e_{i},e_{i} \right\rangle= 0$ for
  $i=1,2$. Alternatively, write $\left\langle x,y \right\rangle = x^{t}\mqty( 0 & 1 \\
  -1 & 0)y$. More generally, $k^{2n}$ can be equipped with a standard
  symplectic form
  \begin{equation}
    \label{eq:14}
  \left\langle x,y \right\rangle =  x^t\mqty(0  & I_n \\ -I_n & 0)y.
  \end{equation}
\end{example}
\begin{exercise}
Show that the dimension of a symplectic space is always even. 
\end{exercise}
\begin{definition}
  Let $W$ be a symplectic space. The \textbf{symplectic group}
  associated to $W$ is the set of linear transformations of $W$
  preserving $\left\langle {-},{-} \right\rangle$:
  \begin{equation}
    \label{eq:6}
  \Sp(W) \defeq \{g \in \GL(W) : \left\langle gx,gy \right\rangle = \left\langle x,y
  \right\rangle \text{ for all } x,y \in W\}.
  \end{equation}
\end{definition}

\begin{exercise}
  Show that $\Sp(k\oplus k) \cong \SL_{2}(k)$, but that $\Sp(W) \neq
  \SL(W)$ in general.
\end{exercise}

\begin{definition}
  A \textbf{Lagrangian subspace} of a symplectic space $W$ is a
  maximal linear subspace $V\subset W$ such that
  $\left\langle v,v' \right\rangle = 0$ for all $v,v'\in V$. A \textbf{Lagrangian
    decomposition}, or \textbf{(principal) polarisation} is a
  decomposition $W = V \oplus V'$ where $V$ and $V'$ are Lagrangian
  subspaces of $W$.
\end{definition}
For example, $W = k\oplus k$ has the rather trivial Lagrangian
decomposition $e_{1}k \oplus e_{2}k$.
\begin{prop}
  Let $W$ be a symplectic space of dimension $2n$. Every Lagrangian
  subspace $ V \subset W$ has dimension $n$, and gives rise to a Lagrangian
  decomposition $W = V \oplus V^{*}$, where $V^{*}$ is the linear dual of $V$.
\end{prop}

\begin{exercise}
  Find a polarisation of $k^{2n}$ with the standard symplectic form.
\end{exercise}

\subsection{The Heisenberg group}
\label{sec:heisenberg-group}
\begin{definition}
  Let $W$ be a symplectic space. The \textbf{Heisenberg group of $W$}
  is the nontrivial central extension of $W$ with underlying set $W \oplus
  k$ and group operation
  \begin{equation}
    \label{eq:8}
(w_{1},t_{2})\cdot (w_{2},t_{2}) \defeq (w_{1} + w_{2}, t_{1}+t_{2} +
\frac{1}{2}\left\langle w_{1},w_{2}\right\rangle ).\footnote{The factor of $1/2$
  is a normalisation thing.}
  \end{equation}
\end{definition}

In other words, $H(W)$ is generally not a direct product of the
underlying groups, but fits into a short exact sequence
\begin{equation}
  \label{eq:8}
0 \to k \to H(W) \to W \to 0,
\end{equation}
which is central: the image of $k$ is precisely the centre of $H(W)$.
If $k = \R$, $H(W)$ is naturally a Lie group, and if $k=\Q_{p}$, a
locally profinite group. Furthermore, it is non-commutative with
$H(W)^{\ab} = W$.

\begin{exercise}
  Describe the Lie algebra of $H(W)$. 
\end{exercise}
Our next goal is to understand the representations of $H(W)$. Let $\psi
\colon k \to \C^{\times}$ be an additive character. A representation $\pi
\colon H(W) \to \GL(S)$ has \emph{central character} $\psi$ if $\pi(t)v =
\psi(t)v$ for all $t \in Z(H(W))$ and $v \in S$. If $\pi$ has trivial central
character, then it factors through $W$.

\begin{theorem}[Stone--von Neumann]\label{thm:SvN}
  Let $\psi\colon k \to \C^{\times}$ be a nontrivial unitary character. Then
  there exists a unique irreducible representation $\omega_{\psi} \colon H(W)
  \to \GL(S)$ with central character $\psi$.
\end{theorem}
This has an explicit description in terms of the \emph{Schrödinger
  model}: fix a polarisation $W = V\oplus V'$, and let $S = \mc S(V)$ be
the set of Schwartz--Bruhat functions on $V$. Then $\omega_{\psi}$ is realised
by
\begin{subequations}
  \begin{align}
    \label{eq:15}
    \omega_{\psi}(0,t)f(x) &= \psi(t)f(x),\\ \omega_{\psi}(v,0)f(x) &= f(x+v),\\ \omega_{\psi}(0,v')f(x) &= \psi(\left\langle x,v' \right\rangle )f(x).
  \end{align}
\end{subequations}
\begin{exercise}
  \begin{enumerate}
  
  \item Check that $\omega_{\psi}$ respects the group operation in $H(W)$,
  \item Think about what happens if you change $\psi$ to $\psi_{a}(x)\defeq
    \psi(ax)$,
  \item Work out the isomorphism of representations obtained by
    interchanging the roles of $V$ and $V'$.
  \end{enumerate}
\end{exercise}
What does this have to do with the symplectic group? For each
$g \in \Sp(W)$, consider the automorphism of $H(W)$ given by
$g\cdot (w,t) = (gw,t)$. Precomposing $\omega_{\psi}$ with this automorphism gives
a new irreducible representation of $H(W)$, hence by \cref{thm:SvN}
these are isomorphic: we can find $M_{g} \in \GL(S)$ such that
$\omega_{\psi}\circ g = M_{g}\omega_{\psi}M_{g}^{-1}$, and by a version of Schur's
lemma\footnote{whose proof we agreed is a bit subtle, because we are
  dealing with infinite-dimensional spaces. There is a proof in
  \href{https://virtualmath1.stanford.edu/\%7Econrad/JLseminar/Notes/L2.pdf}{these notes by Conrad}, says James.} $M_{g}$ is unique up to rescaling by an element of
$\C^{\times}$. This gives rise to a so-called \emph{projective
  representation} of $\Sp W$, $A_{\psi} \to \Sp W \to PGL(S)$.

A well-studied question in representation theory asks ``when can we
lift a projective representation to a genuine representation''? One
way to do so is to pass to a cover: let $\SpC W $ be the closed
subgroup of $\Sp W \times \GL S $ defined by
\begin{equation}
  \label{eq:16}
\SpC_{\psi}(W) \defeq \{(g,M) : \omega_{\psi}\circ g = M \omega_{\psi} M^{-1}\}.
\end{equation}
By what we just discussed, $\SpC_{\psi} (W)$ is a covering of $\Sp(W)$,
fitting into a diagram
\begin{equation}
  \label{eq:18}
% https://tikzcd.yichuanshen.de/#N4Igdg9gJgpgziAXAbVABwnAlgFyxMJZABgBpiBdUkANwEMAbAVxiRAEYQBfU9TXfIRTtyVWoxZsAOlIDCAPRl4AtvG68QGbHgJEy7MfWatEHdX22CiIg9SOTTMhUqyqEPCwN0oATKLsSJiAyAMposgAEAOrmmvw6QsgAzP7ixtJSYdGxWl6JACyp9kGcHnGW3siFtmkOZmW5CUQpNcUZAOIAMgAUIQCUAPROilIqag3xVr6krYEdPf3cYjBQAOZqKKAAZgBOEMpIKSA4EEgAnGW7+0h+x6eISZd7B4iFd+dP14gi7w+fLwBWagnJAADn+SDIv3YEO+wPu+VhbxBiABsKBvwAbLDQfCkAB2WFnPGIcEULhAA
\begin{tikzcd}
1 \arrow[r] & \C^\times \arrow[d] \arrow[r] & \SpC (W) \arrow[d] \arrow[r] & \Sp(W) \arrow[d] \arrow[r]  & 1 \\
1 \arrow[r] & \C^\times \arrow[r]           & \GL(S) \arrow[r]           & \GL(S)/\C^\times \arrow[r] & 1
\end{tikzcd}
\end{equation}
There are other variants of this, for example, taking a toric
extension, as in \cref{sec:intro}.

\begin{prop}
  \begin{enumerate}
  \item $[\SpC_{\psi}(W), \SpC_{\psi}(W)] =: \Mp_{\psi} (W)$ surjects onto
    $\Sp(W)$ with kernel $\{\pm 1\}$.
\item This extension is non-split, and is isomorphic to $\Mp(W)$ which
  is the unique double cover of $\Sp(W)$. 
  \end{enumerate}
\end{prop}
This corresponds to the unique nontrivial cocycle in
$H^{2}(\Sp(W),\{\pm 1\})$. By restricting $\omega_{\psi}$ to $\Mp_{\psi}(W)$, we
get a genuine representation.
\begin{definition}
  The representation $\omega_{\psi} \colon \Mp(W) \to \GL(S)$ is called the
  \textbf{Weil representation} of $W$.
\end{definition}
Note that the Weil representation is unique, at least after fixing the
additive character $\psi$.
\begin{exercise}
Describe the action of $\Sp W$ on $\mc S (W_{1})$ modulo constants.
\end{exercise}
The Weil representation is not irreducible, but splits as $\omega_{\psi} =
\omega_{\psi}^{+} \oplus \omega_{\psi}^{-}$, where the two summands can be described as odd
and even functions, respectively, in the Schrödinger model.

\subsection{What is it good for?}
According to Dan Ciubotaru, a consequence of the double centraliser
theorem is that one should expect simple modules for a pair of
subgroups which are mutual centralisers in a fixed supergroup, to
match up in a certain way. I think this is made precise in
\cite{howe1989}. This motivates:

\begin{definition}
  Let $G$ and $G'$ be algebraic subgroups of $\Sp W$, for some
  symplectic space $W$. We say $(G,G')$ is a \textbf{reductive dual
    pair}  if $Z_{\Sp W }(G) = G'$ and vice versa, and $G$ and $G'$
  are reductive groups.
\end{definition}

\begin{example}
  Let $U$ be a symplectic space, and $V$ an orthogonal space. Then $W
  \defeq U \otimes V$ is naturally a symplectic space with the form defined
  by
  \begin{equation}
    \label{eq:20}
\left\langle u \otimes v, u'\otimes v' \right\rangle_{W} \defeq \left\langle u,u' \right\rangle_{U} \otimes
\left\langle v,v' \right\rangle_{V}, 
  \end{equation}
  and this gives a natural embedding $\Sp U \times O(V) \to \Sp W$.
  Furthermore, it is a good exercise to check that the two factors are
  mutual centralisers in $\Sp W$; in other words, $\Sp U$ and $O(V)$
  form a reductive dual pair.
\end{example}
For finite groups, if $G_{1} \times G_{2} \hookrightarrow G$ and $\pi $ is a representation
of $G$, then $\pi|_{G_{1}\otimes G_{2}} \cong \bigoplus_{i} \pi_{1}^{i}\otimes\pi_{2}^{i}$,
and the hope is that one might be able to understand $\pi_{1}^{i}$ in
terms of $\pi_{2}^{i}$ or vice versa. The hope is that a similar thing
happens for algebraic groups (I believe this is the content of the
Howe duality conjecture, but I'm not quite sure).

For a dual pair $(G,G')$, let $\tilde G$ and $\tilde G'$ denote the
preimages in the metaplectic group. It is then a fact that $\tilde G$
and $\tilde G'$ commute, and we can consider $\omega_{\psi}|_{G\times G'}$.

\begin{conj}[Howe]
  The Weil representation gives a bijection between certain
  representations of $G$ and $G'$: $R_{\psi}(\tilde G) \leftrightarrow R_{\psi}(\tilde G')$.
\end{conj}
We will see a concrete example of this in the next lecture, in the
context of the previous example.
 
\subsection{The global metaplectic group}
\label{sec:glob-metapl-group}

Now let $F$ be a number field, and consider an additive character
$\psi \colon F \backslash \A_{F} \to \C^{\times}$, $\psi = \otimes \psi_{v}$. If $W$ is a symplectic
space over $F$, then localising at $v$ gives a corresponding map of
symplectic spaces $W \to W_{v}/F_{v}$ by extension of scalars. In
particular, for each $v$ we get a metaplectic group $\Mp W_{v}$, and
it is natural to ask if we can glue this to a ``global cover'' of $\Sp
W_{\A_{F}}$.

This is indeed possible. Since we want it to be a double cover
globally as well, let
$Z \defeq \{ (z_{v}) \in \{\pm 1\} : \prod_{v} z_{v} = 1\}$. We define
$\Mp W_{\A_{F}} \defeq \prod_{v}^{'}\Mp(W_{v}) / Z$ with $Z$ coming from
the kernel of each map to $\Sp W_{v}$. Then the
fact that $Z$ acts trivially under $\otimes^{'} \omega_{\psi_{v}}$ implies that the
local representations glue to a big global representation $\Mp
W_{\A_{F}} \colon \omega_{\psi} \to \GL(S)$ where $S$ is an adelic function
space obtained by gluing; one reference is \cite[\S 29]{weil1964}.

Here's a
\href{https://repository.kulib.kyoto-u.ac.jp/dspace/bitstream/2433/101923/1/0727-02.pdf}{reference I found helpful!} In particular, what really matters is not that $G$
and $G'$ are mutual centralisers, but that the von Neumann algebras
generated by the images $\omega (\tilde G)$ and $\omega (\tilde G')$ are mutual
commutators. This is where the connection with Schur--Weyl duality
arises, apparently, but I don't know what that is.


\section{The local Shimura correspondence}
\label{sec:local-shim-corr}
\emph{Speaker: Alex Horawa}

\subsection{Genuine representations}
\label{sec:genu-repr}

Let $k$ be a local field, and suppose $W = ke_{1}\oplus ke_{2}$ is the $2$-dimensional
symplectic space from \cref{eg:standard-sympl}.

\begin{definition}
  An irreducible representation of $\Mp W$ is called \textbf{genuine}
  if it does not factor through $\Sp_{2}(k)$. 
\end{definition}
We denote by $\Irr \Mp W$ the set of such representations.
\begin{remark}
  One might think that the local Langlands correspondence is the right
  tool to understand $\Irr \Mp W$. However, $\Mp W$ is not linear
  algebraic, and so falls outside the domain of the standard Langlands
  conjectures. See however the work of Marty Weissman and others, for example
  \cite{weissman2018}. 
\end{remark}
We have already seen some elements of $\Irr \Mp W$: $\omega_{\psi}^{\pm}$ which
is the even or odd functions in the Schrödinger model of the Weil
representation, is a pair of irreducible representations.
\begin{example}
  Let $a \in k^{\times}$, and consider $\psi_{a} \colon x \mapsto \psi(ax)$ where $\psi$ is
  the usual fixed additive character of $k$. This gives a new
  representation $\omega_{\psi_{a}}$, and one can check that $\omega_{\psi_{a}}\cong
  \omega_{\psi_{b}}$ if and only if $a/b \in (k^{\times})^{2}$. We can also think of
  these as $\omega_{\psi}$ twisted by a quadratic character $\chi : k^{\times} \to \{\pm
  1\}$, as we will soon see. Explicitly, $\omega_{\psi,\chi}^{\pm} = \omega_{\psi_{a}}^{\pm}$
  if $a = \prod b$ where the product runs over $b \in k^{\times}/(k^{\times})^{2}$
  such that $\chi(b) = -1$.
\end{example}

To understand representations of algebraic groups, one often tries to
first understand characters of embedded tori. So, let $T \subset \SL_{2}(k)$
be the diagonal torus $\mqty(a  & 0 \\ 0 & a^{-1})$, and consider the
diagram 
\begin{equation}
  \label{eq:17}
% https://tikzcd.yichuanshen.de/#N4Igdg9gJgpgziAXAbVABwnAlgFyxMJZABgBpiBdUkANwEMAbAVxiRAEYACEAX1PUy58hFO3JVajFmwA6M4HLQBbTuzk9e-EBmx4CRMuwn1mrRB00DdwomKPUT083IUzlq9Ze2C9I5ACZxBykzEDkAWTROAHUvHSF9FED7SVNZGQB3LFg8BlhOABU4nxsUAGZSFMdQor4rBL8KymC05xkAZQAZAH1-TgBrYutE5AAWINSnCzrvYb9xqpC2dl4JGCgAc3giUAAzACcIJSQyEBwIJHYZg6OkQLOLxDLrw+PEMQekUZfbp+pzpAAVh+b0B-0eADYQV9wUgAOzQxBw2GIAAciIhKIAnBiUciznQsAw2AALCAQQaIiqfd7UBhYMChKAQJgAIwYrGoJJgdCgbEgjJA-0JxPMAtYiLBNPGBKJpPJlIoPCAA
\begin{tikzcd}
1  \arrow[r] & \{\pm 1\} \arrow[r]                                & \Mp W \arrow[r]                        & \SL_2 (k) \arrow[r]           & 1 \\
1 \arrow[r]  & \{\pm 1\} \arrow[r] \arrow[u, Rightarrow, no head ] & \widetilde T \arrow[r] \arrow[u, hook] & T \arrow[r] \arrow[u, hook] & 1
\end{tikzcd}
\end{equation}
Here $\widetilde T$ is the preimage of $T$ in $\Mp W$ as usual.

\begin{definition}
  We denote by $\chi_{\psi} \colon \widetilde T \to \C^{\times}$ the character
  sending $(t(a),\epsilon)$ to $\epsilon \cdot \gamma(a,\psi)^{-1}$, where $\gamma(a,\psi)$ is the
  epsilon-factor $\epsilon(1/2,\eta_{a}, \bar \psi)$, $\eta_{a}$ being the quadratic
  character associated to the extension $k(\sqrt a) /k$. 
\end{definition}
One can also write $\gamma(a,\psi) = \gamma(\psi_{a})/\gamma(\psi)$ where the latter $\gamma$ is
the so-called \emph{Weil index}. One reference for this equality is
\cite{szpruch2018}.

\begin{prop}\label{prop:torus-char}
  The map $\mu \mapsto \chi_{\psi}\cdot \mu$ gives a bijection between characters of $T$
  and genuine characters of $\widetilde T$.
\end{prop}
Note that this bijection depends on a choice of character $\psi$. Now let
$Z \subset T$ be the centre of $\SL_{2}(k)$ and consider its lift
$\widetilde Z$. Depending on whether or not $-1$ is a square in $k$,
this is either isomorphic to $C_{2}\times C_{2}$ or $C_{4}$. If $\sigma \in \Irr
\Mp W$, then the central character is either $\chi_{\psi}|_{\tilde Z}$ or
$\chi_{\psi}|_{\tilde Z} \cdot \sgn$. Let $z_{\psi}(\sigma)$ be $1$ or $-1$ in the
corresponding cases.

Now we want to describe $\Irr \Mp W$ using the theta correspondence.
Let $(V,q)$ be a $3$-dimensional orthogonal space. As before, we have
a dual reductive pair $\Mp W \times O(V) \subset \Mp W\otimes V$. The Bruhat--Schwartz
space $\mc S(ke_{2}\otimes V)$ can naturally be identified with $\mc S
\defeq \mc S(V)$, and we have the following explicit formulas for the
Weil representation:\footnote{NB: there was some confusion during the
  lecture, and these formulas might not be correct. One paper that's
  often cited with presumably correct formulas is \cite{rao1993}.}
 \begin{subequations}
  \begin{align}
    \omega_{\psi}(\epsilon,h) \cdot \phi (v)&= \phi(h^{-1}v) \qq{for} h \in O(V),\\
    \omega_{\psi}(\epsilon,t(a))\cdot \phi(v) &= |a|^{3/2}\chi_{\psi}(a)\phi(v) \qq{for} t(a) =
                          \mqty(a  & 0 \\ 0 & a^{-1}),\\
    \omega_{\psi}(\epsilon, n(x)) \cdot \phi(v) &= \psi(x\cdot q(v))\cdot \phi(v) \qq{for} n(x) = \mqty(1
                        & x \\ 0 & 1),\label{eq:weil-unipot}\\
    \omega_{\psi}(\epsilon, \mqty(0  & 1 \\ -1 & 0))\phi(v) &= \gamma(\psi) \int_{V} \psi(-\left\langle v,z \right\rangle \phi(z))dz.
  \end{align}
\end{subequations}

Next we define theta lifts on the level of representations:

\begin{definition}\label{def:big-theta}
  The \textbf{big $\Theta$-lift} of a smooth $O(V)$ representation $\pi$ is
  the $\pi$-isotypic component of $\omega_{\psi,V,W}|_{O(V)\times\Mp W}$. In other words, it is the
  largest smooth representation $\Theta(\pi) = \Theta_{\psi,V,W}(\pi)$ of $\Mp W$ such
  that $\pi \boxtimes \Theta(\pi) \subset \omega_{\psi,V,W}$.
\end{definition}
This gives a map $\Rep O(V) \to \Rep \Mp W$. Completely analogously, we
define a big $\Theta$-lift in the reverse direction. To understand this map
explicitly, we first need an concrete understanding of orthogonal
spaces of dimension $3$.
\begin{prop}
  Let $V/k$ be a $3$-dimensional orthogonal space. Then $V$ is either
  $V^{+}$ satisfying $\SO(V^{+}) \cong \PGL_{2}(k)$, or $V^{-}$ satisfying
  $\SO(V^{-}) \cong PD^{\times}$ where $D/k$ is the unique nonsplit quaternion
  algebra over $k$.
\end{prop}
We distinguish these by the invariant $\epsilon(V^{\pm})  \defeq \pm 1$.
\subsection{Waldspurger's theorem}
\label{sec:waldspurgers-theorem}

\begin{theorem}[Waldspurger]\mbox{\ }
  \begin{enumerate}
  \item If $\pi \in \Irr \SO(V)$, then
    \begin{enumerate}
    \item there exists a unique character $\epsilon$ such that $\Theta_{\psi}(\pi^{\epsilon})
      \ne \emptyset$,
    \item For this $\epsilon$, $\Theta_{\psi}(\pi^{\epsilon})$ has a unique irreducible
      quotient $\vartheta_{\psi}(\pi^{\epsilon}) \in \Irr \Mp W$,
    \item $\epsilon = \epsilon(V) \cdot \epsilon(1/2,\pi,\psi)$.
    \end{enumerate}
  \item If $\sigma \in \Irr \Mp W$, then 
    \begin{enumerate}
    \item there exists a unique $3$-dimensional orthogonal space $V$
      such that $\Theta_{\psi}(\sigma) \ne 0$, 
    \item $\Theta_{\psi}(\sigma)$ has a unique irreducible quotient $\vartheta_{\psi}(\sigma)$,
    \item $\epsilon(V) = z_{\psi}(\sigma) \cdot \epsilon(1/2,\sigma,\psi)$, where the last factor is
      some local $\epsilon$-factor for $\Mp W$ which we have yet to define.
    \end{enumerate}
  \item The map
    \begin{equation}
      \label{eq:19}
\Irr \Mp W \to \Irr \SO(V^{+}) \sqcup \Irr \SO(V^{-}) \qq{given by} \sigma \mapsto \vartheta_{\psi}(\sigma)
    \end{equation}
is a bijection preserving $L$, $\epsilon$ and $\gamma$-factors, suitably defined
for $\Mp W$.
  \end{enumerate}
\end{theorem}
The bijection in \cref{eq:19} is called the \textit{Shimura
  correspondence}, and the maps $\vartheta$ are called \emph{small
  theta lifts}.

There is an alternative way to package these results: by the
Jacquet--Langlands correspondence, we can match up
representations of $\SO(V^{-})$ with discrete series representations
of $\SO(V^{+})$. If $D$ denotes the associated quaternion algebra over
$k$, let $\pi_{D}$ be a representation of $PD^{\times}$, and $\pi \defeq
\operatorname{JL} (\pi_{D})$ be the Jacquet--Langlands lift. Then we can 
rewrite Waldspurger's theorem as an equality
\begin{equation}
  \label{eq:21}
\Irr \Mp W = \bigsqcup_{\pi \in \Irr \PGL_{2}(k)}A_{\psi}(\pi) \qq{where}
A_{\psi}(\pi) = \{\sigma^{+} \defeq \theta_{\psi,W,V^{+}}(\pi), \ \sigma^{-} \defeq \theta_{\psi,W,V^{-}}(\pi_{D})\},
\end{equation}
where $\pi_{D}$ is assumed trivial if $\pi$ is not discrete series. The
set $A_{\psi}(\pi)$ is called a \emph{Waldspurger packet}.
\begin{remark}
  Analogous ideas were used by Howe and Piatetski-Shapiro to cook up a
  counterexample to the naive generalisation of the Ramanujan
  conjecture to arbitrary automorphic forms. For them $V$ is an
  orthogonal space of real signature $(2,2)$, giving a lift to $\Sp
  4$. 
\end{remark}

\subsection{An explicit local Shimura correspondence}
\label{sec:an-explicit-local}

Let's try to make this even more explicit, using the classification of
representations of $\PGL_{2}(k)$: Let $B = \mqty(*  & * \\ 0 & *)$ be
the upper-triangular Borel subgroup, and $\mu \colon k^{\times} \to \C^{\times}$ a
character. Then $\mu \times \mu^{-}$ is evaluated at the diagonal entries is
naturally a character of $B$, and define $\pi(\mu,\mu^{-1}) \defeq
\Ind_{B}^{G}\mu \times \mu^{-1}$.

\begin{prop}
  The following is a classification of irreducible representations of
  $\PGL_{2}(k)$: 
  \begin{enumerate}
  \item The representation $\pi(\mu,\mu^{-1})$ is irreducible if and only if $\mu^{2}
\ne |\cdot|^{\pm1}$. 
\item If $\mu = \chi \cdot |\cdot |^{1/2}$ for some quadratic character $\chi$, then
  there is a short exact sequence
  \begin{equation}
    \label{eq:22}
0 \to \St_{\chi} \to \pi(\mu, \mu^{-1}) \to \chi \circ \det \to 0,
  \end{equation}
where $\St_{\chi}$ is an irreducible representation of $\PGL_{2}(k)$
called the \emph{twist of the Steinberg representation}. Similarly, if
$\mu = \chi \cdot |\cdot|^{-1/2}$, we have a short exact sequence
\begin{equation}
  \label{eq:23}
  0 \to \chi \circ \det \to \pi(\mu,\mu^{-1}) \to \St_{\chi} \to 0
\end{equation}
(is this also twist of Steinberg? Confused.)
\item The representations not covered by this construction are known
  as \emph{supercuspidal representations}. 
  \end{enumerate}
\end{prop}

The representation theory of $PD^{\times}$ is simpler. The numbering is
meant to match up with the previous proposition under Jacquet--Langlands.

\begin{prop}
  The irreducible representations of $PD^{\times}$ are as follows:
  \begin{enumerate}
  \item $\emptyset$ --- there are no principal series representations. 
  \item $\chi\circ N_{D}$ where $N_{D} \colon D^{\times} \to k^{\times}$ is the norm
    character.
  \item irreducible representations of dimension $>1$.
  \end{enumerate}
\end{prop}
Finally, we need to describe the representations of the metaplectic
group.

\begin{prop}
  The irreducible representations of $\Mp W$ for $\dim W = 2$ are
  given by:

  \begin{enumerate}
  \item let $\mu$ be a character of $T$ so that $\mu \cdot \chi_{\psi}$ is a
    character of $\tilde T$, giving the \emph{principal series
      representation} $\sigma_{\psi}(\mu) = \Ind_{\tilde B}^{\Mp
      W} \mu\cdot \chi_{\psi}$\footnote{Not obvious how this gives a character of
      the Borel?}. This is irreducible iff $\mu^{2} \ne |\cdot|^{\pm 1}$. 
  \item If $\mu = \chi \cdot |\cdot |^{1/2}$ with $\chi$ quadratic, then we get a
    \emph{twisted Steinberg representation}  by
    \begin{equation}
      \label{eq:24}
0 \to \St_{\psi,\chi} \to \sigma_{\psi,\mu} \to \omega_{\psi,\chi}^{+}\to 0,
\end{equation}
and similar for $\mu = \chi \cdot |\cdot |^{-1/2}$. 
\item Supercuspidal representations, for example $\omega^{-}_{\psi}$. 
  \end{enumerate}
\end{prop}

We can now match up the representations explicitly:

\begin{table}[h]\label{tab:sh-corr}
  \centering
  \begin{tabular}{c|c|c}
    $\Irr \Mp W$ & $\Irr \PGL_{2}(k)$ & $\Irr PD^{\times}$ \\
    \hline\hline $\sigma_{\psi}(\mu)$& $\pi(\mu,\mu^{-1})$ & \\
    $\St_{\psi,\1}$\cellcolor{skog!25}  & & $\1$\cellcolor{skog!25} \\
    $\St_{\psi,\chi}, \chi \neq \1$\cellcolor{hav!15}  & \cellcolor{hav!15} $\St_{\chi}$ & \\
    $\omega_{\psi,\chi}^{+}$ & $\chi \circ \det$ & \\
    $\omega_{\psi}^{-}$ \cellcolor{skog!25}  & $\St_{\1}$\cellcolor{skog!25} & \\
    $\omega_{\psi,\chi}^{-}$\cellcolor{hav!15}  & & $\chi \circ N_{D}$\cellcolor{hav!15}  \\ 
    Other sc. & Other sc. & Other sc.
  \end{tabular}
  \caption{Explicit Shimura correspondence}
  \label{tab:shimura}
\end{table}
Here we have marked in similar colours the representations appearing
in the same Waldspurger packet.


\section{The global Shimura correspondence}
\label{sec:glob-shim-corr}
\emph{Speaker: Alex Horawa}

\subsection{Global Waldspurger packets}
\label{sec:glob-waldsp-pack}

Today we will describe irreducible discrete series representations of
$\Mp W_{\A}$ in terms of those of $\PGL_{2}\A$. Suppose $\sigma$ is an
automorphic representation of $\Mp W_{\A}$. A version of Flath's
theorem for the metaplectic group then implies that $\sigma = \otimes_{v}\sigma_{v}$,
where each $\sigma_{v}$ lies in the Waldspurger packet of some
representation $\pi_{v}$ of $\PGL_{2} F_{v}$: $\sigma_{v} \in A_{\psi_{v}} (\pi_{v})$ 

We want to construct \emph{global} Waldspurger packet, and for this we
need to understand which $\sigma_{v}$ ``glue together''.

\begin{definition}
  Let $\pi$ be an automorphic representation of $\PGL_{2}\A$. The
  \textbf{global Waldspurger packet} attached to $\pi$ is the set
  \begin{equation}
    \label{eq:25}
\{ \sigma^{\epsilon} \defeq \otimes_{v}' \sigma_{v}^{\epsilon_{v}} : \epsilon \in \otimes_{v} \{\pm 1\} \text{ and }
\sigma_{v}^{\epsilon_{v}} \in A_{\psi_{v}}(\pi_{v})\},
  \end{equation}
where the sign $\epsilon_{v}$ attached to $\sigma_{v}^{\epsilon_{v}}$ matches the sign in
the local packet $A_{\psi_{v}}(\pi_{v})$.
\end{definition}
Now $\sigma^{\epsilon}$ is an abstract representation of $\Mp W_{\A}$, and we ask:
\begin{enumerate}
\item Is $\sigma^{\epsilon}$ always an automorphic representation?
\item Does every automorphic representation of $\Mp W_{\A}$ arise in
  this way?
\end{enumerate}
It turns out that the answer to both of these questions is ``no''.
For example, we recall from \cref{tab:sh-corr} that $ \omega_{\psi_{v}, \chi_{v}}^{-} \in
A_{\psi_{v}}(\chi_{v}\circ \det)$ for $\chi_{v}$ nontrivial. But $\chi \circ \det$ is a
one-dimensional representation, so the local components do not arise 
from an irreducible representation of the metaplectic group.

Now construct $\omega_{\psi} = \otimes'_{v} \omega_{\psi_{v}} \colon \Mp W_{\A} \to \mc
S(V_{\A})$. Since $\omega_{\psi_{v}} = \omega_{\psi_{v}}^{-} \oplus \omega_{\psi_{v}}^{+}$, this is
``highly reducible'' in the sense that each local component is
reducible. For a fixed finite set of places $S$ of $F$ we define 
\begin{equation}
  \label{eq:26}
\omega_{\psi,S}
\defeq \big(\otimes_{v \in S} \omega_{\psi_{v}}^{-}\big) \otimes \big(\otimes_{v \notin S} \omega_{\psi_{v}}^{+}\big),
\end{equation}
and we claim that these have a better chance of being automorphic.

\begin{prop}
  Let $\Theta_{\psi}$ be the big theta lift from \cref{def:big-theta}, and
  identify its domain with $\mc S(\A)$:
  \begin{equation}
    \label{eq:27}
\Theta_{\psi} \colon \phi \mapsto \qty(g \mapsto \sum_{x \in F} (\omega_{\psi}(g)\phi )(x)).
  \end{equation}

  \begin{enumerate}
  \item For any $\phi$, $\Theta_{\psi}(\phi)$ is left $\Sp W_{F}$-invariant,
  \item $\ker \Theta_{\psi} = \bigoplus_{\# S \text{ odd}}\omega_{\psi,S}$, so that
    $\im \Theta_{\psi} \cong \bigoplus_{\# S \text{ even}} \omega_{\psi,S}$,
  \item $\im \Theta_{\psi,S} \subset L^{2}_{\disc}(\Mp W_{\A})$, and the image is
    cuspidal unless $S = \emptyset$.
 \end{enumerate}
\end{prop}
We can also obtain a version of this after twisting by a quadratic character
$\chi\colon F^{\times}\backslash \A^{\times} \to \C$; this amounts to replacing $\psi$ with $\psi_{a}$. We define $\Theta_{\psi,\chi}$
and $\omega_{\psi,\chi}$ in the natural way. Pick a nontrivial set of places $S$,
and consider $\omega_{\psi,\chi,S}$, which one shows is an automorphic
representation not coming from a global Waldspurger packet. However,
we have the following:

\begin{equation}
  \label{eq:28}
L_{\disc}^{2}(\Mp W_{\A}) \subset \qty(\bigcup_{S} \omega_{\psi,\chi,S})\cup \qty(\bigcup_{\pi} A_{\psi}(\pi)),
\end{equation}
so $\omega_{\psi,\chi,S}$ is essentially the only counterexample.
\begin{prop}
  Let $\sigma^{\epsilon}$ be as above. If $\sigma^{\epsilon}$ is nontrivial, then $\prod_{v} \epsilon_{v}
  = \epsilon(1/2,\pi,\psi)$. 
\end{prop}
In fact, the converse is also true, as we will show in the remainder
of the seminar.
\begin{proof}
  By the discussion after \cref{prop:torus-char}, the central character of $\sigma_{v}^{\epsilon_{v}}$ is given by
  $\omega_{\sigma_{v}^{\epsilon_{v}}}/\chi_{\psi_{v}} = \epsilon_{v} \cdot \epsilon(1/2,\pi_{v},\psi_{v})$, viewed
  as characters of $Z(F_{v}) = \{\pm1\}$. But $\chi_{\psi}  = \oplus_{v} \chi_{\psi_{v}}$
  and $\omega_{\sigma^{\epsilon}}$ global automorphic characters restricted from
  $\tilde T$ to $Z(\A)$. Their
  product is trivial, so $1 = (\prod_{v} \epsilon_{v}) \cdot \epsilon(1/2,\pi,\psi)$, and this
  proves the claim.
\end{proof}

\subsection{The main global theorem}
\label{sec:main-global-theorem}
\begin{theorem}[Main global theorem]
  We have a decomposition
  \begin{equation}
    \label{eq:29}
    L^{2}_{\disc}(\Mp W_{\A}) = \underbrace{\qty[\bigoplus_{\chi
        {\text{quad.}}}\bigoplus_{\#S \in 2\Z_{\ge 0}} \omega_{\psi,\chi,S}]}_{\text{elementary
        theta functions}}\oplus \bigoplus_{\pi \in \mc A_{0}(\PGL_{2}(\A))}L^{2}_{\pi},
  \end{equation}
where $L_{\pi}^{2} = \bigoplus_{\sigma^{\epsilon}} m_{\psi}(\sigma^{\epsilon})\cdot \sigma^{\epsilon}$, the sum
running over $\sigma^{\epsilon} \in \mc A_{\psi}(\pi)$, and
\begin{equation}
  \label{eq:29}
m_{\psi}(\sigma^{\epsilon}) \defeq \begin{cases}
                     1 \qq{if} \prod_{v} \epsilon_{v} = \epsilon(1/2,\pi,\psi), \\ 0 \qq{otherwise.}
                    \end{cases}
\end{equation}
\end{theorem}
Furthermore, each $L^{2}_{\pi}$ is a ``full near equivalence class''.
We will spend the rest of the course trying to understand the proof of
this theorem. First, we need to construct $\sigma^{\epsilon}$ explicitly. To that
end, we revert to the language of a orthosymplectic dual pair
$O(V) \times \Mp W \subset \Mp W\otimes V$, and use the Weil representation:

\begin{definition}
  Let $\pi$ be an automorphic representation of $O(V)$. For any
  algebraic group $G$ over $F$, we write
  $[G] \defeq G(F)\backslash G(\A)$. The \textbf{theta lift of $\pi$}, $\Theta_{\psi}(\pi)$ is the
  representation spanned by
  \begin{equation}
    \label{eq:30}
\theta_{\psi,W,V}(\phi,f) \colon \Sp W_{F} \backslash \Mp W_{\A} \to \C, \qq{} \theta_{\psi,W,V}(\phi,f)  \colon g \mapsto \int_{[O(V)]} \theta_{\psi}(\phi)(g,h)\overline{f(h)}dh,
  \end{equation}
  as $\phi$ runs through elements of $\mc S(V_{\A})$, and
  $f \in \pi$. Here $\theta_{\psi}$ is the global theta lift
  on $\Mp W_{\A}\times O(V_{\A})$,
\begin{equation}
  \label{eq:32}
\theta(\phi)(g,h) = \sum_{x \in V_{F}} (\omega_{\psi}(g,h)\phi)(x),
\end{equation}
for $\omega_{\psi}$ the global Weil representation on $\Mp W_{\A} \times O(V_{\A})$.
\end{definition}
As usual, we suppress subscripts which are clear from the context.
One should think of $\theta(\phi,f)$ as the ``$f$-isotypic component of
$\theta(\phi)(g,h)$'', by orthogonality of the inner product. This one could
plausible hope to be an automorphic representation --- assuming
convergence --- and which could be cuspidal when $\pi$ is. This will be
shown in the next lecture.

We will take the following as a black box:
\begin{prop}
  The representation $\Theta_{\psi,W,V}(\pi)$ has an irreducible quotient
  isomorphic to $\otimes_{v}' \theta_{\psi,W_{v},V_{v}}(\pi_{v})$, with local components
  defined as in \cref{sec:waldspurgers-theorem}.
\end{prop}

\section{Periods of global theta lifts}
\label{sec:periods-global}
\emph{Speaker: Håvard Damm-Johnsen}

In this lecture, we will try to understand the representation
$\Theta_{\psi}(\pi)$ by computing its associated \emph{period functionals}. This
will allow us to show prove non-vanishing and cuspidality.

\subsection[O(V) vs SO(V)]{$O(V)$ vs $\SO(V)$}
\label{sec:OV-vs-SOV}

In calculations, it is often easier to work with $\SO(V)$ than $O(V)$.
We claim that little is lost in doing so: first, note that
$O(V) = SO(V) \times \mu_{2}$, where $\mu_{2}$ is the algebraic group with
underlying set $\{\pm1\}$. An automorphic representation of $\mu_{2}$ is
of the form
\begin{equation}
  \label{eq:34}
\sgn_{S} \colon [\mu_{2}] \to \C \qq{} \sgn_{S} = \qty(\bigotimes_{v \in S}
\sgn_{v})\otimes \qty(\bigotimes_{v \notin S} \1_{v}),
\end{equation}
where $S$ is some finite set of places of $F$, and any representation
$\pi$ of $\SO(V)$ can be extended to $O(V)$ by fixing some $S$ and
taking $\pi \boxtimes \sgn_{S}$. Note that $S$ should be even for
$\sgn_{S}$ to factor through the adelic quotient. We can
extend the definition of $\theta_{\psi}$ by setting $\Theta_{\psi}(\pi) = \sum_{\# S \in 2
  \Z_{\ge 0}} \Theta_{\psi}(\pi \boxtimes \sgn_{S})$. However, the following proposition
shows that it is not necessary when $\pi$ is cuspidal:

\begin{prop}
  Let $\pi$ be a cuspidal automorphic representation of $\SO(V)$.
  \begin{enumerate}
  \item There is at most one set $S$ such that $\Theta_{\psi}(\pi \boxtimes \sgn_{S})$
    is nonzero.
  \item This set $S$ is characterised by the property
\begin{equation}
  \label{eq:31}
v \in S \qq{if and only if} \epsilon(1/2,\pi_{v}, \psi_{v}) = -1.
\end{equation}
  \end{enumerate}
\end{prop}
This is a consequence of the main local theorem as stated in Gan's
lectures, Thm.~2.1. A consequence is that if $\# S$ is odd, then
$\Theta_{\psi}(\pi) = 0$. The content of the proposition is therefore both a
vanishing criterion, \emph{and} the statement that passing to $\SO(V)$
gives no loss of generality.

\subsection{Periods}
\label{sec:periods}
Modular forms gives rise to numerical invariants known as periods
by integrating against suitable sets, and these generalise to the
adelic setting:
\begin{definition}
  Let $G$ be a reductive algebraic group, $H \le G$ a reductive subgroup,
  and $\chi \colon [H] \to \C$ a character of $H$. If $\pi$ is an automorphic
  representation of $G$ and $f \in \pi$, define the
  \textbf{$(H,\chi)$-period of $f$} to be the \emph{function}
  \begin{equation}
    \label{eq:33}
f_{H,\chi}(g) \defeq \int_{H}f(hg)\overline{\chi(h)}dh.
  \end{equation}
The \textbf{$(H,\chi)$-period functional} $\mc P_{H,\chi}$ is the linear map
$\pi \to \C$ sending $f$ to $f_{H,\chi}(1)$.
\end{definition}
Note that $\mc P_{H,\chi}(\pi) \ne 0$ implies that $\pi$ is nontrivial; if so,
we call $\pi$ ``$(H,\chi)$-distinguished''.

\begin{example}
  Let $\tilde f$ be a modular form of weight $k$ and level
  $\Gamma_{0}(N)$, and $f$ its associated automorphic form:
  $\tilde f(z) = f(g_{z}) \cdot y^{-k/2}$, where
  $g_{z} = (\Id_{2},\ldots, (\mqty(y & x \\ 0 & 1))) \in \GL_{2}(\A)$. Let
  $H = \mqty(1 & * \\ 0 & 1)$, and for $n \in \Z$ let
  $\chi = \otimes_{v}\chi_{v}$ where
  $\chi_{\infty}(x) = e^{2\pi i nx}$ and
  $\chi_{p}(x) = e^{-2\pi in \{x\}}$.\footnote{Curly braces means
    ``$p$-adic fractional part''.} Here and below we identify $H$ with
  $[\A] = \Q \backslash \A$ via the map
  $x \mapsto \mqty(1 & x \\ 0 & 1)$. Then we compute
\begin{subequations}
  \begin{align}
    \label{eq:35}
    f_{H,\chi}(g_{z})
    &= \int_{[\A]}f(\mqty(1  & x \\ 0 & 1)g_{z})\chi(x)dx \\
    &= \prod_{p}\qty(\int_{\Z_{p}}f(\mqty(1  & x_{p} \\ 0 & 1))e^{2\pi in
                                                     \{x_{p}\}}dx_{p})\cdot
                                                     \int_{\R/\Z}
                                                     f(g_{z+x_{\infty}})e^{-2\pi
                                                     in x_{\infty}}dx_{\infty}\\
    &= 1\cdot \int_0^{1} \tilde f(z+x)y^{k/2}e^{-2\pi i nx_{\infty}}dx_{\infty} =
      a_{n}(\tilde f)y^{k/2}.
  \end{align}
\end{subequations}
In other words, the $(H,\chi)$-period computes exactly the $n$-th Fourier
coefficient of $f$, and being $(H,\chi)$ distinguished for this choice of
$H$ simply means that the corresponding Fourier coefficient does not
vanish.
\end{example}
In the example, we see that when $\chi$ is trivial and
$N = \mqty(1 & * \\ 0 & 1)$, then $f_{N,\chi}$ is precisely the constant
term of $\tilde f$. Now we want to compute this for automorphic forms
on $\Mp W_{\A}$, recalling that $N$ splits over $\Mp W_{\A}$:
\begin{subequations}
  \begin{align}
    \label{eq:36}
    \mc P_{N}(\theta_{\psi}(\phi,f))
    & = \int_{[N]}\theta_{\psi}(\phi,f)(n)dn = \int_{[N]}\int_{[\SO(V)]}
      \theta(\phi)(n,h)\overline{f(h)}dhdn \\
    & = \int_{[N]}\int_{[\SO(V)]}\overline{f(h)} \qty(\sum_{x \in V_{F}}
      (\omega_{\psi}(n,h)\phi)(x))dhdn =: (*)
  \end{align}
\end{subequations}
Now, a suitably altered version of \cref{eq:weil-unipot} implies that
$\omega_{\psi}(n,h)\phi (x) = \psi(n q(x))\phi(h^{-1}x)$, and this gives
\begin{equation}
  \label{eq:38}
(*) = \int_{[SO(V)]}\overline{f(h)}\qty(\sum_{x \in V_{F}}\phi(h^{-1}x)\int_{[N]}
\psi(n q(x))dn)dh = \int_{[SO(V)]} \overline{f(h)}\qty(\sum_{\substack{x \in
    V_{F} \\ q(x) = 0 }} \phi(h^{-1}x))dh
\end{equation}
where the last equality comes from orthogonality of characters.

We now split into two cases: either $V$ is a \emph{definite}
orthogonal space, meaning $q(x) = 0$ if and only if $x = 0$, or not.
If it is, then $\mc P_{N}(\phi,f) = \phi(0) \int_{[SO(V)]}\overline{f(h)}dh$,
which we recognise as a multiple of the constant term of $\bar f$. In
particular, this is zero iff $f$ is a cusp form.

The case of $V$ indefinite is a bit more complicated. Recall that we
call a nontrivial element $x\in V_{F}$ \emph{isotropic} if $q(x) = 0$.
It is a fact that $\SO(V)$ acts transitively on isotropic
vectors\footnote{I think one way to see this is to use
  \cite[Thm.~42:17]{omeara1963}: argue that the subspaces spanned by two
  isotropic vectors are isometric, and that this isometry extends to
  all of $V$. In fact, this argument shows that $O(V)$ acts
  transitively on $d$-dimensional isotropic subspaces for each $d$.
  Why the isometry has determinant $1$ requires some more thought.},
and the stabiliser $U$ of a fixed isotropic vector $x_{0}$ is the 
unipotent radical of a Borel of $\SO(V)$. [Find reference to Conrad]
\begin{example}
  Let $B = M_{2}(\Q)$ and $V = B^{\tr = 0}$ so that $\SO(V) =
  \PGL_{2}(\Q)$, acting on $V$ by conjugation. The vector $x_{0} = \mqty(0  & 1
  \\ 0 & 0)$ is isotropic, and we have $\Stab_{\PGL_{2}(\Q)} x_{0} =
  \mqty(1  & * \\ 0 & 1)$, which is indeed a unipotent subgroup of the
  upper triangular Borel.
\end{example}
It follows that $X_{0} \defeq \{ x \in V_{F} : q(x) = 0 \} = \{0\} \sqcup
\SO(V)/ U \cdot x_{0}$. Now
\begin{subequations}
  \begin{align}
    \label{eq:39}
    \mc P_{N}(\theta_{\psi}(\phi,f))
    &= \int_{[\SO(V)]} \overline{f(h)}\qty(\sum_{\gamma \in U(F)\backslash
      \SO(V_{F})}\phi(h^{-1}\gamma^{-1}x_{0}))dh \\
    &= \int_{U(F)\backslash \SO(V_{\A})}\overline{f(h)}\phi(h^{-1}x_{0})dh \\
    &= \int_{(U\backslash\SO(V))(\A)}\int_{[U]}\overline{f(uh)}\phi(h^{-1}u^{-1}x_{0})dudh \\
    &= \int_{(U\backslash\SO(V))(\A)}\phi(h^{-1}x_{0})\underbrace{\qty(\int_{[U]}
      \overline{f(uh)}du)}_{\text{period of $\bar f$}}dh
  \end{align}
\end{subequations}
The inner integral in the last equation is more or less the constant
term of $f$, hence we have:

\begin{prop}
  Let $\pi$ be a cuspidal automorphic representation of $\SO(V)$. The
  representation $\Theta_{\psi}(\pi)$ is cuspidal unless $V$ is anistropic and
  $\pi$ trivial.
\end{prop}

While we have found a criterion for when $\Theta_{\psi}(\pi)$ is cuspidal, we
still don't know if it is non-zero. To characterise this, we will exhibit non-zero
periods, or Fourier coefficients. Fix a character $\psi_{a}$ of $N$ with
$a \in F^{\times}$. By an identical argument as above, for any $f\in \pi$ we have
\begin{equation}
  \label{eq:40}
\mc P_{N,\psi_{a}}(f) = \int_{[\SO(V)]}\overline{f(h)}\sum_{x \in
  X_{a}(F)}\phi(h^{-1}x) dh
\end{equation}
where $X_{a}(F) = \{x \in V_{F} : q(x) = a\}$. We fix $x_{a} \in
X_{a}(F)$, and by Witt's extension theorem, the
decomposition $V = Fx_{a} \oplus V_{a}$ makes $V_{a} = (Fx_{a})^{\perp}$ a
quadratic space; furthermore, $\Stab_{\SO(V)} x_{a} = \SO(V_{a})$,
which is an embedded torus in $\SO(V)$.
\begin{example}
Take $x_{a} = \mqty(1  & 0 \\ 0 & -1)$ in $M_{2}(\Q)$, with $q(x_{a})
= -\det x_{a} = 1$. Then $V_{a} = \Q \mqty(0  & 1 \\ 0 & 0) \oplus \Q
\mqty(0  & 0 \\ 1 & 0)$, which is an isotropic subspace, and one
checks that $\Stab_{\PGL_{2}(\Q)}x_{a}$ is the diagonal
torus. This can be identified with $\SO(1,1) = \SO(V_{a})$.
\end{example}
With the same manipulations as for $a = 0$, we get
\begin{equation}
  \label{eq:41}
\mc P_{N,\psi_{a}}(\theta_{\psi}(\phi,f)) = \int_{(\SO(V_{a})\backslash
  \SO(V))(\A)}\phi(h^{-1}x_{a}) \underbrace{\int_{[\SO(V_{a})]}
  \overline{f(th)}dt}_{=: \mc{P}_{V_{a}}(h\cdot \bar f)}dh.
\end{equation}
So the nonvanishing of the the period $\mc P_{N,\psi_{a}}(\theta_{\psi}(\phi,f))$
implies the nonvanishing of the torus period $\mc P_{V_{a}}(f)$. With
a little more work, one can show the converse.

\begin{prop}
  Let $\pi$ be a cuspidal automorphic representation of $\SO(V)$. The
  $(N,\psi_{a})$-period of $\Theta_{\psi}(\pi)$ is trivial if and only if $\mc
  P_{V_{a}}$ is non-zero on $\pi$. 
\end{prop}

\subsection{Split torus periods}
\label{sec:split-torus-periods}
Now we specialise further, to the case where $B = M_{2}(F)$, and
$T_{a} \subset PB^{\times}$ is the split diagonal torus, $T_{a} = \mqty(*  & 0 \\
0 & 1)$. Then we can relate the toric period functional $\mc
P_{V_{a}}$ from the previous section, to central $L$-values of twists
of $\pi$ using the following theorem:

\begin{prop}[Hecke--Jacquet--Langlands]
  Let
  \begin{equation}
    \label{eq:42}
    Z(s,f) \defeq \int_{[\A^{\times}]} f \mqty(a  & 0 \\ 0 & 1) |a|^{s-1/2}da,
  \end{equation}
  and let $S$ be the set of places of $F$ which are ramified for
  $\pi$. Then for any decomposable $f = \otimes_{v}f_{v} \in \pi$ we have $Z(s,f)
  = L(s,\pi) \cdot \prod_{v \in S}Z_{v}^{\#}(s,f_{v})$, where
  $Z_{v}^{\#}(s,f_{v})$ is a local zeta integral. Furthermore, for any
  $s_{0}$ there exists $f_{v} \in \pi_{v}$ such that
  $Z^{\#}_{v}(s_{0},f_{v}) \ne 0$.
\end{prop}
\begin{cor}
  The period functional $\mc P_{T_{a}}$ is non-zero on $\pi$ if and only
  if $L(1/2,\pi) \ne 0$. In particular, nonvanishing implies $\Theta_{\psi}(\pi) \ne
  0$, independently of $\psi$.
\end{cor}
In fact, there's no loss of generality in studying split torus
periods as opposed to all of them:

\begin{theorem}
  Let $\pi$ be a cuspidal automorphic representation of $\SO(V) \cong
  \PGL_{2}(F)$. Then the following are equivalent:
  \begin{enumerate}
  \item $L(1/2,\pi) \ne 0$, 
  \item $\pi$ has a non-zero split torus period,
  \item $\pi$ has a non-zero torus period,
  \item $\Theta_{\psi}(\pi) \ne 0$.
  \end{enumerate}
\end{theorem}

Now that we have two invariants depending on $\pi$ with equal
``support'', it is natural to ask whether they are proportional.
Waldspurger proved that this is indeed the case:

\begin{theorem}[Waldspurger]
  Let $\pi$ be a cuspidal automorphic representation of $\PGL_{2}(F)$,
  and let $\pi^{\vee}$ denote its contragredient. For any $f_{1} \in \pi $ and
  $f_{2} \in \pi^{\vee}$ and any torus $T \subset \PGL_{2}(F)$, we have
  \begin{equation}
    \label{eq:37}
\mc P_{T,\chi}(f_{1})\mc P_{T,\bar\chi} (f_{2}) = c \frac{L(1/2, \pi\otimes
  \chi)}{L(1,\pi, \Ad)} \cdot (\text{local factors}),
  \end{equation}
where $c$ is some simple, explicit constant not depending on $\pi$.
\end{theorem}


\printbibliography%
\end{document}


%%% Local Variables:
%%% mode: latex
%%% TeX-master: t
%%% End:
